\chapter{Trace and Monitoring Support}
\label{chap:trace}

The procedures and functions described in this chapter can be found in
\textasciitilde\emph{ns}/{trace.\{cc, h\}},
\textasciitilde\emph{ns}/{tcl/lib/ns-trace.tcl},
\textasciitilde\emph{ns}/{queue-monitor.\{cc, h\}},
\textasciitilde\emph{ns}/{tcl/lib/ns-link.tcl},
\textasciitilde\emph{ns}/{packet.h},
\textasciitilde\emph{ns}/{flowmon.cc}, and
\textasciitilde\emph{ns}/{classifier-hash.cc}.

There are a number of ways of collecting output or
trace data on a simulation.
Generally, trace data is either displayed directly during execution
of the simulation, or (more commonly) stored in a file to be
post-processed and analyzed.
There are two primary but distinct types of monitoring capabilities
currently supported by the simulator.
The first, called {\em traces}, record each individual packet
as it arrives, departs, or is dropped at a link or queue.
Trace objects are configured into a simulation as nodes in the
network topology, usually with a Tcl ``Channel'' object
hooked to them, representing the destination of collected data
(typically a trace file in the current directory).
The other types of objects, called {\em monitors}, record counts
of various interesting quantities such as packet and byte arrivals,
departures, etc.
Monitors can monitor counts associated with all packets,
or on a per-flow basis using a 
{\em flow monitor} below (Section~\ref{sec:flowmon}).

To support traces, there is a special {\em common} header
included in each packet (this format is defined in \textasciitilde\emph{ns}/{packet.h}
as {\tt hdr\_cmn}).
It presently includes a unique identifier on each packet, a
packet type field (set by agents when they generate packets),
a packet size field (in bytes, used to determine the transmission
time for packets), and an interface label (used for computing
multicast distribution trees).

Monitors are supported by a separate
set of objects that are created and inserted into the network topology
around queues.
They provide a place where
arrival statistics and times are gathered and make use of the
\clsref{Integrator}{../ns-2/integrator.h} (Section~\ref{sec:integral})
to compute statistics over time intervals.

\section{Trace Support}
\label{sec:otcltrace}

The trace support in OTcl consists of a number of specialized
classes visible in OTcl but implemented in C++, combined
with a set of Tcl helper procedures and classes defined in the \emph{ns}\ library.

All following OTcl classes are supported by underlying C++
classes defined in \textasciitilde\emph{ns}/{trace.cc}.
Objects of the following types are inserted directly in-line in the
network topology:

\begin{tabularx}{\linewidth}{rX}
Trace/Hop & trace a ``hop'' (XXX what does this mean exactly; it is not really used XXX) \\
Trace/Enque & a packet arrival (usually at a queue) \\
Trace/Deque & a packet departure (usually at a queue) \\
Trace/Drop & packet drop (packet delivered to drop-target) \\
Trace/Recv & packet receive event at the destination node of a link \\
SnoopQueue/In & on input, collect a time/size sample (pass packet on) \\
SnoopQueue/Out & on output, collect a time/size sample (pass packet on) \\
SnoopQueue/Drop & on drop, collect a time/size sample (pass packet on) \\
SnoopQueue/EDrop & on an "early" drop, collect a time/size sample (pass packet on) \\
\end{tabularx}

Objects of the following types are added in the simulation and a referenced
by the objects listed above.  They are used to aggregate statistics collected
by the SnoopQueue objects:

\begin{tabularx}{\linewidth}{rX}
QueueMonitor & receive and aggregate collected samples from snoopers \\
QueueMonitor/ED & queue-monitor capable of distinguishing between ``early'' and standard packet drops \\
QueueMonitor/ED/Flowmon & per-flow statistics monitor (manager) \\
QueueMonitor/ED/Flow & per-flow statistics container \\
QueueMonitor/Compat & a replacement for a standard QueueMonitor when \emph{ns}~v1 compatibility is in use \\
\end{tabularx}

\subsection{OTcl Helper Functions}
\label{sec:helptrace}

The following helper functions may be used within simulation
scripts to help in attaching trace elements (see \textasciitilde\emph{ns}/{tcl/lib/ns-lib.tcl});
they are instance procedures of the class Simulator:

\begin{tabularx}{\linewidth}{rX}
{\tt flush-trace \{\}} & flush buffers
        for all trace objects in simulation \\
{\tt create-trace \{ type file src dst \}} & create
        a trace object of type {\em type}
        between the given src and dest nodes.
        If {\em file} is non-null, it is interpreted as a Tcl channel
        and is attached to the newly-created trace object. \
        The procedure returns the handle to the newly created trace object.\\
{\tt trace-queue \{ n1 n2 file \}} & arrange for tracing on the link
        between nodes {\em n1} and {\em n2}.
        This function calls create-trace,
        so the same rules apply with respect to the {\em file} argument. \\
{\tt trace-callback\{ ns command \}} & arranges to call {\tt command}
        when a line is to be traced.
        The procedure treats {\tt command}
        as a string and evaluates it for every line traced.
        See \textasciitilde\emph{ns}/{tcl/ex/callback\_demo.tcl} for additional details on usage.\\
{\tt monitor-queue \{ n1 n2 \}} & this function
        calls the {\tt init-monitor} function
        on the link between nodes {\em n1} and {\em n2}. \\
{\tt drop-trace \{ n1 n2 trace \}} & the given {\em trace} object
        is made the drop-target of the queue
        associated with the link between nodes {\em n1} and {\em n2}.
\end{tabularx}

The \proc[]{create-trace} procedure is used to create a new {\tt Trace}
object of the appropriate kind and attach an Tcl I/O channel to it
(typically a file handle).
The {\tt src\\_} and {\tt dst\\_} fields are are used by the underlying C++
object for producing the trace output file so that trace output
can include the node addresses defining the endpoints of the link which
is being traced.
Note that they are not used for {\em matching}.  Specifically, these
values in no way relate to the packet header {\tt src} and {\tt dst}
fields, which are also displayed when tracing.
See the description of the {\tt Trace} class below (Section~\ref{sec:tracemoncplus}).

The {\tt trace-queue} function enables
{\tt Enque}, {\tt Deque}, and {\tt Drop} tracing on the link
between nodes {\tt n1} and {\tt n2}.
The Link {\tt trace} procedure is described below (Section~\ref{sec:libexam}).

The {\tt monitor-queue} function is constructed similarly to
{\tt trace-queue}.
By calling the link's {\tt init-monitor} procedure, it arranges
for the creation of objects ({\tt SnoopQueue} and {\tt QueueMonitor}
objects) which can, in turn, be used to ascertain time-aggregated
queue statistics.

The {\tt drop-trace} function provides a way to specify a
{\tt Queue}'s drop target without having a direct handle of
the queue.

\section{Library support and examples}
\label{sec:libexam}

The {\tt Simulator} procedures described above require the {\tt trace}
and {\tt init-monitor} methods associated with the OTcl {\tt Link} class.
Several subclasses of link are defined, the most common of which
is called {\tt SimpleLink}.  Thus, the {\tt trace} and {\tt init-monitor}
methods are actually part of the {\tt SimpleLink} class rather than
the {\tt Link} base class.
The {\tt trace} function is defined as follows (in {\tt ns-link.tcl}):
\begin{verbatim}
#
# Build trace objects for this link and
# update the object linkage
#
SimpleLink instproc trace { ns f } {
        \$self instvar enqT\_ deqT\_ drpT\_ queue\_ link\_ head\_ fromNode\_ toNode\_
        \$self instvar drophead\_

        set enqT\_ [\$ns create-trace Enque \$f \$fromNode\_ \$toNode\_]
        set deqT\_ [\$ns create-trace Deque \$f \$fromNode\_ \$toNode\_]
        set drpT\_ [\$ns create-trace Drop \$f \$fromNode\_ \$toNode\_]

        \$drpT\_ target [\$drophead\_ target]
        \$drophead\_ target \$drpT\_
        \$queue\_ drop-target \$drpT\_

        \$deqT\_ target [\$queue\_ target]
        \$queue\_ target \$deqT\_

        if { [\$head\_ info class] == "networkinterface" } {
            \$enqT\_ target [\$head\_ target]
            \$head\_ target \$enqT\_
            # puts "head is i/f"
        } else {
            \$enqT\_ target \$head\_
            set head\_ \$enqT\_
            # puts "head is not i/f"
        }
}
\end{verbatim}
This function establishes {\tt Enque}, {\tt Deque}, and {\tt Drop}
traces in the simulator {\tt \$ns} and directs their
output to I/O handle {\tt \$f}.
The function assumes a queue has been associated with the link.
It operates by first creating three new trace objects
and inserting the {\tt Enque} object before the queue, the
{\tt Deque} object after the queue, and the {\tt Drop} object
between the queue and its previous drop target.
Note that all trace output is directed to the same I/O handle.

This function performs one other  additional tasks.
It checks to see if a link contains a network interface,
and if so, leaves it as the first object in the chain of objects
in the link, but otherwise inserts the {\tt Enque} object as
the first one.

The following functions, {\tt init-monitor} and
{\tt attach-monitor}, are used to create a set of
objects used to monitor queue sizes of a queue associated
with a link.
They are defined as follows:
\begin{verbatim}
        SimpleLink instproc attach-monitors { insnoop outsnoop dropsnoop qmon } {
                \$self instvar queue\_ head\_ snoopIn\_ snoopOut\_ snoopDrop\_
                \$self instvar drophead\_ qMonitor\_

                set snoopIn\_ \$insnoop
                set snoopOut\_ \$outsnoop
                set snoopDrop\_ \$dropsnoop

                \$snoopIn\_ target \$head\_
                set head\_ \$snoopIn\_

                \$snoopOut\_ target [\$queue\_ target]
                \$queue\_ target \$snoopOut\_

                \$snoopDrop\_ target [\$drophead\_ target]
                \$drophead\_ target \$snoopDrop\_

                \$snoopIn\_ set-monitor \$qmon
                \$snoopOut\_ set-monitor \$qmon
                \$snoopDrop\_ set-monitor \$qmon
                set qMonitor\_ \$qmon
        }

        #
        # Insert objects that allow us to monitor the queue size
        # of this link.  Return the name of the object that
        # can be queried to determine the average queue size.
        #
        SimpleLink instproc init-monitor { ns qtrace sampleInterval} {
                \$self instvar qMonitor\_ ns\_ qtrace\_ sampleInterval\_

                set ns\_ \$ns
                set qtrace\_ \$qtrace
                set sampleInterval\_ \$sampleInterval
                set qMonitor\_ [new QueueMonitor]

                \$self attach-monitors [new SnoopQueue/In] \bs
                        [new SnoopQueue/Out] [new SnoopQueue/Drop] \$qMonitor\_

                set bytesInt\_ [new Integrator]
                \$qMonitor\_ set-bytes-integrator \$bytesInt\_
                set pktsInt\_ [new Integrator]
                \$qMonitor\_ set-pkts-integrator \$pktsInt\_
                return \$qMonitor\_
        }
\end{verbatim}
These functions establish queue monitoring on the {\tt SimpleLink} object
in the simulator {\tt ns}.
Queue monitoring is implemented by constructing three {\tt SnoopQueue}
objects and one {\tt QueueMonitor} object.
The {\tt SnoopQueue} objects are linked in around a {\tt Queue} in a way
similar to {\tt Trace} objects.
The {\tt SnoopQueue/In(Out)} object monitors packet arrivals(departures)
and reports them to an associated {\tt QueueMonitor} agent.
In addition, a {\tt SnoopQueue/Out} object is also used to accumulate
packet drop statistics to an associated {\tt QueueMonitor} object.
For {\tt init-monitor} the same {\tt QueueMonitor} object is used
in all cases.
The C++ definitions of the {\tt SnoopQueue} and {\tt QueueMonitor}
classes are described below.

\section{The C++ Trace Class}
\label{sec:tracemoncplus}

Underlying C++ objects are created in support of the interface specified
in Section~\ref{sec:tracemoncplus} and are linked into the network topology
as network elements.
The single C++ {\tt Trace} class is used to implement the OTcl
classes {\tt Trace/Hop}, {\tt Trace/Enque}, {\tt Trace/Deque},
and {\tt Trace/Drop}.
The {\tt type\\_} field is used to differentiate among the
various types of
traces any particular {\tt Trace} object might implement.
Currently, this field may contain one of the following symbolic characters:
{\bf +} for enque, {\bf -} for deque, {\bf h} for hop, and
{\bf d} for drop.
The overall class is defined as follows in \textasciitilde\emph{ns}/{trace.cc}:
\begin{verbatim}
        class Trace : public Connector {
         protected:
                int type\_;
                nsaddr\_t src\_;
                nsaddr\_t dst\_;
                Tcl\_Channel channel\_;
                int callback\_;
                char wrk\_[256];
                void format(int tt, int s, int d, Packet* p);
                void annotate(const char* s);
                int show\_tcphdr\_;  // bool flags; backward compat
         public:
                Trace(int type);
                ~Trace();
                int command(int argc, const char*const* argv);
                void recv(Packet* p, Handler*);
                void dump();
                inline char* buffer() { return (wrk\_); }
        };
\end{verbatim}
The {\tt src\\_}, and {\tt dst\\_} internal state is used
to label trace output and is independent of the corresponding field
names in packet headers.
The main \fcn[]{recv} method is defined as follows:
\begin{verbatim}
        void Trace::recv(Packet* p, Handler* h)
        {
                format(type\_, src\_, dst\_, p);
                dump();
                /* hack: if trace object not attached to anything free packet */
                if (target\_ == 0)
                        Packet::free(p);
                else
                        send(p, h); /* \fcn[]{Connector::send} */
        }
\end{verbatim}
The function merely formats a trace entry using the source, destination,
and particular trace type character.
The {\tt dump} function writes the formatted entry out to the
I/O handle associated with {\tt channel\\_}.
The {\tt format} function, in effect, dictates the trace file format.

\section{Trace File Format}
\label{sec:traceformat}

The \fcn[]{Trace::format} method defines the trace file format used
in trace files produced by the {\tt Trace} class.
It is constructed to maintain backward compatibility with output files
in earlier versions of the simulator (\ie, \emph{ns}~v1) so that \emph{ns}~v1
post-processing scripts continue to operate.
The important pieces of its implementation are as follows:
\begin{verbatim}
        // this function should retain some backward-compatibility, so that
        // scripts don't break.
        void Trace::format(int tt, int s, int d, Packet* p)
        {
                hdr\_cmn *th = (hdr\_cmn*)p-\>access(off\_cmn\_);
                hdr\_ip *iph = (hdr\_ip*)p-\>access(off\_ip\_);
                hdr\_tcp *tcph = (hdr\_tcp*)p-\>access(off\_tcp\_);
                hdr\_rtp *rh = (hdr\_rtp*)p-\>access(off\_rtp\_);
                packet\_t t = th-\>ptype();
                const char* name = packet\_info.name(t);

                if (name == 0)
                        abort();

                int seqno;
                /* XXX */
                /* CBR's now have seqno's too */
                if (t == PT\_RTP || t == PT\_CBR)
                        seqno = rh-\>seqno();
                else if (t == PT\_TCP || t == PT\_ACK)
                        seqno = tcph-\>seqno();
                else
                        seqno = -1;

                \ldots

                if (!show\_tcphdr\_) {
                        sprintf(wrk\_, "%c %g %d %d %s %d %s %d %d.%d %d.%d %d %d",
                                tt,
                                Scheduler::instance().clock(),
                                s,
                                d,
                                name,
                                th-\>size(),
                                flags,
                                iph-\>flowid() /* was p-\>class\_ */,
                                iph-\>src() \>\> 8, iph-\>src() & 0xff,     // XXX
                                iph-\>dst() \>\> 8, iph-\>dst() & 0xff,     // XXX
                                seqno,
                                th-\>uid() /* was p-\>uid\_ */);
                } else {
                        sprintf(wrk\_,
                        "%c %g %d %d %s %d %s %d %d.%d %d.%d %d %d %d 0x%x %d",
                                tt,
                                Scheduler::instance().clock(),
                                s,
                                d,
                                name,
                                th-\>size(),
                                flags,
                                iph-\>flowid() /* was p-\>class\_ */,
                                iph-\>src() \>\> 8, iph-\>src() & 0xff,     // XXX
                                iph-\>dst() \>\> 8, iph-\>dst() & 0xff,     // XXX
                                seqno,
                                th-\>uid(), /* was p-\>uid\_ */
                                tcph-\>ackno(),
                                tcph-\>flags(),
                                tcph-\>hlen());
                }
\end{verbatim}
This function is somewhat unelegant, primarily due to the desire
to maintain backward compatibility.
It formats the source, destination, and type fields defined in the
trace object ({\em not in the packet headers}), the current time,
along with various packet header fields including,
type of packet (as a name), size, flags (symbolically),
flow identifier, source and destination packet header fields,
sequence number (if present), and unique identifier.
The {\tt show\_tcphdr\_} variable indicates whether the trace
output should append tcp header information (ack number, flags, header length)
at the end of each output line.
This is especially useful for simulations using
FullTCP agents (Section~\ref{sec:fulltcp}).
An example of a trace file (without the tcp header fields) might
appear as follows: 
\begin{small}
\begin{verbatim}
+ 1.84375 0 2 cbr 210 ------- 0 0.0 3.1 225 610
- 1.84375 0 2 cbr 210 ------- 0 0.0 3.1 225 610
r 1.84471 2 1 cbr 210 ------- 1 3.0 1.0 195 600
r 1.84566 2 0 ack 40 ------- 2 3.2 0.1 82 602
+ 1.84566 0 2 tcp 1000 ------- 2 0.1 3.2 102 611
- 1.84566 0 2 tcp 1000 ------- 2 0.1 3.2 102 611
r 1.84609 0 2 cbr 210 ------- 0 0.0 3.1 225 610
+ 1.84609 2 3 cbr 210 ------- 0 0.0 3.1 225 610
d 1.84609 2 3 cbr 210 ------- 0 0.0 3.1 225 610
- 1.8461 2 3 cbr 210 ------- 0 0.0 3.1 192 511
r 1.84612 3 2 cbr 210 ------- 1 3.0 1.0 196 603
+ 1.84612 2 1 cbr 210 ------- 1 3.0 1.0 196 603
- 1.84612 2 1 cbr 210 ------- 1 3.0 1.0 196 603
+ 1.84625 3 2 cbr 210 ------- 1 3.0 1.0 199 612
\end{verbatim}
\end{small}
Here we see 14 trace entries, five enque operations (indicated by ``+''
in the first column), four deque operations (indicated by ``-''),
four receive events (indicated by ``r''), and one drop event.
(this had better be a trace fragment, or
some packets would have just vanished!).
The simulated time (in seconds) at which each event occurred is listed
in the second column.
The next two fields indicate between which two nodes tracing is happening.
The next field is 
a descriptive name for the the type of packet seen 
(Section~\ref{sec:traceptype}).
The next field is the packet's size, as encoded in its IP header.

The next field contains the flags, which not used in this example.
The flags are defined in the flags[] array in trace.cc. 
Four of the flags are used for ECN:
``E'' for Congestion Experienced (CE) and ``N'' for 
ECN-Capable-Transport (ECT) indications
in the IP header, and 
``C'' for ECN-Echo and ``A'' for Congestion Window Reduced (CWR) 
in the TCP header.
For the other flags, ``P'' is for priority, and ``F'' is for TCP Fast Start.

The next field gives the IP {\em flow identifier} field as defined
for IP version 6.\footnote{In \emph{ns}~v1, each packet included a \code{class}
field, which was used by CBQ to classify packets.
It then found additional use to differentiate between
``flows'' at one trace point.  In \emph{ns}~v2, the flow ID field is available
for this purpose, but any additional information (which was commonly overloaded
into the class field in \emph{ns}~v1) should be placed in its own separate field,
possibly in some other header}.
The subsequent two fields indicate the packet's source and destination
node addresses, respectively.
The following field indicates the sequence number.\footnote{In \emph{ns}~v1,
all packets contained a sequence number, whereas in \emph{ns}~v2 only those
Agents interested in providing sequencing will generate sequence numbers.
Thus, this field may not be useful in \emph{ns}~v2 for packets generated by
agents that have not filled in a sequence number.  It is used here
to remain backward compatible with \emph{ns}~v1.}
The last field is a unique packet identifier.  Each new packet
created in the simulation is assigned a new, unique identifier.

\section{Packet Types}
\label{sec:traceptype}

Each packet contains a packet type field used by {\tt Trace::format}
to print out the type of packet encountered.
The type field is defined in the {\tt TraceHeader} class, and is considered
to be part of the trace support; it is not interpreted
elsewhere in the simulator.
Initialization of the type field in packets is performed by the
\fcn{Agent::allocpkt} method.
The type field is set to integer values associated with the
definition passed to
the {\tt Agent} constructor (Section~\ref{sec:agents:exlinkage}).
The currently-supported definitions, their values, and their
associated symblic names are as follows
(defined in \textasciitilde\emph{ns}/{packet.h}):
\begin{verbatim}
enum packet\_t {
	PT\_TCP,
	PT\_UDP,
	PT\_CBR,
	PT\_AUDIO,
	PT\_VIDEO,
	PT\_ACK,
	PT\_START,
	PT\_STOP,
	PT\_PRUNE,
	PT\_GRAFT,
	PT\_GRAFTACK,
	PT\_JOIN,
	PT\_ASSERT,
	PT\_MESSAGE,
	PT\_RTCP,
	PT\_RTP,
	PT\_RTPROTO\_DV,
	PT\_CtrMcast\_Encap,
	PT\_CtrMcast\_Decap,
	PT\_SRM,
	/* simple signalling messages */
	PT\_REQUEST,	
	PT\_ACCEPT,	
	PT\_CONFIRM,	
	PT\_TEARDOWN,	
	PT\_LIVE,	// packet from live network
	PT\_REJECT,

	PT\_TELNET,	// not needed: telnet use TCP
	PT\_FTP,
	PT\_PARETO,
	PT\_EXP,
	PT\_INVAL,
	PT\_HTTP,
	/* new encapsulator */
	PT\_ENCAPSULATED,
	PT\_MFTP,
	/* CMU/Monarch's extnsions */
	PT\_ARP,
	PT\_MAC,
	PT\_TORA,
	PT\_DSR,
	PT\_AODV,
	PT\_AOMDV,
        /* DCCP */
        PT\_DCCP,
        ...
        PT\_MDART,

	// insert new packet types here

	PT\_NTYPE // This MUST be the LAST one
};
\end{verbatim}
The constructor of class {\tt p\_info} glues these constants with
their string values:
\begin{verbatim}
	p\_info() {
		name\_[PT\_TCP]= "tcp";
		name\_[PT\_UDP]= "udp";
		name\_[PT\_CBR]= "cbr";
		name\_[PT\_AUDIO]= "audio";
		...
		name\_[PT\_NTYPE]= "undefined";
	}
\end{verbatim}
See also section~\ref{sec:pinfoclass} for more details.

\section{Queue Monitoring}
\label{sec:qmonitor}

Queue monitoring refers to the capability of tracking the
dynamics of packets at a queue (or other object).
A queue monitor tracks packet arrival/departure/drop statistics,
and may optionally compute averages of these values.
Monitoring may be applied to all packets (aggregate statistics), or
per-flow statistics (using a Flow Monitor).

Several classes are used in supporting queue monitoring.
When a packet arrives at a link where queue monitoring is enabled,
it generally passes through a {\tt SnoopQueue} object when it
arrives and leaves (or is dropped).
These objects contain a reference to a {\tt QueueMonitor} object.

A {\tt QueueMonitor} is defined as follows (\textasciitilde\emph{ns}/{queue-monitor.cc}):
\begin{verbatim}
        class QueueMonitor : public TclObject {
         public: 
                QueueMonitor() : bytesInt\_(NULL), pktsInt\_(NULL), delaySamp\_(NULL),
                  size\_(0), pkts\_(0),
                  parrivals\_(0), barrivals\_(0),
                  pdepartures\_(0), bdepartures\_(0),
                  pdrops\_(0), bdrops\_(0),
                  srcId\_(0), dstId\_(0), channel\_(0) {

                        bind("size\_", &size\_);
                        bind("pkts\_", &pkts\_);
                        bind("parrivals\_", &parrivals\_);
                        bind("barrivals\_", &barrivals\_);
                        bind("pdepartures\_", &pdepartures\_);
                        bind("bdepartures\_", &bdepartures\_);
                        bind("pdrops\_", &pdrops\_);
                        bind("bdrops\_", &bdrops\_);
                        bind("off\_cmn\_", &off\_cmn\_);
                };

                int size() const { return (size\_); }
                int pkts() const { return (pkts\_); }
                int parrivals() const { return (parrivals\_); }
                int barrivals() const { return (barrivals\_); }
                int pdepartures() const { return (pdepartures\_); }
                int bdepartures() const { return (bdepartures\_); }
                int pdrops() const { return (pdrops\_); }
                int bdrops() const { return (bdrops\_); }
                void printStats();
                virtual void in(Packet*);
                virtual void out(Packet*);
                virtual void drop(Packet*);
                virtual void edrop(Packet*) { abort(); }; // not here
                virtual int command(int argc, const char*const* argv);
                \ldots

        // packet arrival to a queue
        void QueueMonitor::in(Packet* p)
        {
                hdr\_cmn* hdr = (hdr\_cmn*)p-\>access(off\_cmn\_);
                double now = Scheduler::instance().clock();
                int pktsz = hdr-\>size();

                barrivals\_ += pktsz;
                parrivals\_++;
                size\_ += pktsz;
                pkts\_++;
                if (bytesInt\_)
                        bytesInt\_-\>newPoint(now, double(size\_));
                if (pktsInt\_)
                        pktsInt\_-\>newPoint(now, double(pkts\_));
                if (delaySamp\_)
                        hdr-\>timestamp() = now;
                if (channel\_)
                        printStats();
        }

        \ldots \fcn[]{in}, \fcn[]{out}, \fcn[]{drop} are all defined similarly \ldots
\end{verbatim}
It addition to the packet and byte counters, a queue monitor
may optionally refer to objects that keep an integral
of the queue size over time using
{\tt Integrator} objects, which are defined in Section~\ref{sec:integral}.
The {\tt Integrator} class provides a simple implementation of
integral approximation by discrete sums.

All bound variables beginning with {\bf p} refer to packet counts, and
all variables beginning with {\bf b} refer to byte counts.
The variable {\tt size\_} records the instantaneous queue size in bytes,
and the variable {\tt pkts\_} records the same value in packets.
When a {\tt QueueMonitor} is configured to include the integral
functions (on bytes or packets or both), it
computes the approximate integral of the
queue size (in bytes)
with respect to time over the interval $[t_0, now]$, where
$t_0$ is either the start of the simulation or the last time the
{\tt sum\\_} field of the underlying {\tt Integrator} class was reset.

The {\tt QueueMonitor} class is not derived from {\tt Connector}, and
is not linked directly into the network topology.
Rather, objects of the {\tt SnoopQueue} class (or its derived classes)
are inserted into the network topology, and these objects contain references
to an associated queue monitor.
Ordinarily, multiple {\tt SnoopQueue} objects will refer to the same
queue monitor.
Objects constructed out of these classes are linked in the simulation
topology as described above and call {\tt QueueMonitor}
{\tt out}, {\tt in}, or {\tt drop} procedures,
depending on the particular type of snoopy queue.

\section{Per-Flow Monitoring}
\label{sec:flowmon}

A collection of specialized classes are used to to implement
per-flow statistics gathering.
These classes include: \\
{\tt QueueMonitor/ED/Flowmon},
{\tt QueueMonitor/ED/Flow}, and {\tt Classifier/Hash}.
Typically, an arriving packet is inspected to determine
to which flow it belongs.
This inspection and flow mapping is performed by a {\em classifier}
object (described in section~\ref{sec:flowmonitor}).
Once the correct flow is determined, the packet is passed to
a {\em flow monitor}, which is responsible for collecting per-flow
state.
Per-flow state is contained in {\em flow} objects in a one-to-one
relationship to the flows known by the flow monitor.
Typically, a flow monitor will create flow objects on-demand when
packets arrive that cannot be mapped to an already-known flow.

\subsection{The Flow Monitor}
\label{sec:flowmonitor}

The {\tt QueueMonitor/ED/Flowmon} class is responsible for managing
the creation of new flow objects when packets arrive on previously
unknown flows and for updating existing flow objects.
Because it is a subclass of {\tt QueueMonitor}, each flow monitor
contains an aggregate count of packet and byte arrivals, departures, and
drops.
Thus, it is not necessary to create a separate queue monitor to record
aggregate statistics.
It provides the following OTcl interface:
\begin{quote}
\begin{tabularx}{\linewidth}{rX}
         classifier & get(set) classifier to map packets to flows\\
         attach & attach a Tcl I/O channel to this monitor\\
         dump  & dump contents of flow monitor to Tcl channel\\
         flows & return string of flow object names known to this monitor\\
\end{tabularx}
\end{quote}

The {\tt classifier} function sets or gets the name of the previously-allocated
object which will perform packet-to-flow mapping for the flow monitor.
Typically, the type of classifier used will have to do with the notion of
``flow'' held by the user.
One of the hash based classifiers that inspect various IP-level header
fields is typically used here (e.g. fid, src/dst, src/dst/fid).
Note that while classifiers usually receive packets and forward them
on to downstream objects, the flow monitor uses the classifier only for
its packet mapping capability, so the flow monitor acts as a passive
monitor only and does not actively forward packets.

The {\tt attach} and {\tt dump} functions are used to
associate a Tcl I/O stream with the
flow monitor, and dump its contents on-demand.
The file format used by the {\tt dump} command is described below.

The {\tt flows} function returns a list of the names of flows known
by the flow monitor in a way understandable to Tcl.
This allows tcl code to interrogate a flow monitor in order
to obtain handles to the individual flows it maintains.

\subsection{Flow Monitor Trace Format}
\label{sec:flowmonclass}

The flow monitor defines a trace format which may be used by post-processing
scripts to determine various counts on a per-flow basis.
The format is defined by the following code in \textasciitilde\emph{ns}/{flowmon.cc}:
\begin{verbatim}
void
FlowMon::fformat(Flow* f)
{   
        double now = Scheduler::instance().clock();
        sprintf(wrk\_, "%8.3f %d %d %d %d %d %d %d %d %d %d %d %d %d %d %d %d %d 
                now,    
                f-\>flowid(),    // flowid
                0,              // category
                f-\>ptype(),     // type (from common header) 
                f-\>flowid(),    // flowid (formerly class)
                f-\>src(),
                f-\>dst(),
                f-\>parrivals(), // arrivals this flow (pkts)
                f-\>barrivals(), // arrivals this flow (bytes) 
                f-\>epdrops(),   // early drops this flow (pkts)
                f-\>ebdrops(),   // early drops this flow (bytes) 
                parrivals(),    // all arrivals (pkts)
                barrivals(),    // all arrivals (bytes) 
                epdrops(),      // total early drops (pkts)
                ebdrops(),      // total early drops (bytes) 
                pdrops(),       // total drops (pkts)
                bdrops(),       // total drops (bytes) 
                f-\>pdrops(),    // drops this flow (pkts) [includes edrops] 
                f-\>bdrops()     // drops this flow (bytes) [includes edrops]
        );
};  
\end{verbatim}
Most of the fields are explained in the code comments.
The ``category'' is historical, but is used to maintain loose backward-
compatibility with the flow manager format in \emph{ns}~version 1.

\subsection{The Flow Class}
\label{sec:flowclass}

The class {\tt QueueMonitor/ED/Flow} is used by the flow monitor
for containing per-flow counters.
As a subclass of {\tt QueueMonitor}, it inherits the standard
counters for arrivals, departures, and drops, both in packets and
bytes.
In addition, because each flow is typically identified by
some combination of the packet source, destination, and flow
identifier fields, these objects contain such fields.
Its OTcl interface contains only bound variables:
\begin{quote}
\begin{\par\tabular{\textwidth}{rX}}
        {\tt src\\_} &   source address on packets for this flow\\
        {\tt dst\\_} &   destination address on packets for this flow\\
        {\tt flowid\\_} & flow id on packets for this flow\\
\end{\par\tabular{\textwidth}{rX}}
\end{quote}

Note that packets may be mapped to flows (by classifiers) using
criteria other than a src/dst/flowid triple.
In such circumstances, only those fields actually used by
the classifier in performing the packet-flow mapping should be
considered reliable.


\section{Commands at a glance}
\label{sec:tracecommand}

Following is a list of trace related commands commonly used in
simulation scripts:
\begin{flushleft}
{\tt \$ns\_ trace-all \<tracefile\>}\\
This is the command used to setup tracing in ns. All traces are written in
the <tracefile>.


{\tt \$ns\_ namtrace-all \<namtracefile\>}\\
This command sets up nam tracing in ns. All nam traces are written in to
the <namtracefile>.


{\tt \$ns\_ namtrace-all-wireless \<namtracefile\> \<X\> \<Y\>}\\
This command sets up wireless nam tracing. <X> and <Y> are the x-y co-ordinates
for the wireless topology and all wireless nam traces are written  into
the <namtracefile>.


{\tt \$ns\_ nam-end-wireless \<stoptime\>}\\
This tells nam the simulation stop time  given in <stoptime>.


{\tt \$ns\_ trace-all-satlinks \<tracefile\>}\\
This is a method to trace satellite links and write traces into <tracefile>.


{\tt \$ns\_ flush-trace}\\
This command flushes the trace buffer and is typically called before the
simulation run ends.


{\tt \$ns\_ get-nam-traceall}\\
Returns the namtrace file descriptor stored as the Simulator instance
variable called {\tt namtraceAllFile\_}.


{\tt \$ns\_ get-ns-traceall}\\
Similar to get-nam-traceall. This returns the file descriptor for ns tracefile
which is stored as the Simulator instance called {\tt traceAllFile\_}.


{\tt \$ns\_ create-trace \<type\> \<file\> \<src\> \<dst\> \<optional:op\>}\\
This command creates a trace object of type <type> between the <src> and
<dst> nodes. The traces are written into the <file>. <op> is the argument
that may be used to specify the type of trace, like nam. if <op> is not
defined, the default trace object created is for nstraces.


{\tt \$ns\_ trace-queue \<n1\> \<n2\> \<optional:file\>}\\
This is a wrapper method for {\tt create-trace}. This command creates a
trace object for tracing events on the link represented by the nodes <n1>
and <n2>.


{\tt \$ns\_ namtrace-queue \<n1\> \<n2\> \<optional:file\>}\\
This is used to create a trace object for namtracing on the link between
nodes <n1> and <n2>. This method is very similar to and is the namtrace
counterpart of method {\tt trace-queue}.


{\tt \$ns\_ drop-trace \<n1\> \<n2\> \<trace\>}\\
This command makes the given <trace> object a drop-target for the queue
associated with the link between nodes <n1> and <n2>.


{\tt \$ns\_ monitor-queue \<n1\> \<n2\> \<qtrace\> \<optional:sampleinterval\>}\\
This sets up a monitor that keeps track of average queue length of the queue
on the link between nodes <n1> and <n2>. The default value of
sampleinterval is 0.1. 


{\tt \$link trace-dynamics \<ns\> \<fileID\> }
Trace the dynamics of this link and write the output to fileID filehandle.
ns is an instance of the Simulator or MultiSim object that was created to
invoke the simulation. 

The tracefile format is backward compatible with the output files in the
ns version 1 simulator so that ns-1 postprocessing scripts can still be
used. Trace records of traffic for link objects with Enque, Deque, receive
or Drop Tracing have the following form: \\

<code> <time> <hsrc> <hdst> <packet> \\

where 
\begin{verbatim}
<code> := [hd+-] h=hop d=drop +=enque -=deque r=receive <time> :=
simulation time in seconds 
<hsrc> := first node address of hop/queuing link 
<hdst> := second node address of hop/queuing link 
<packet> := <type> <size> <flags> <flowID> <src.sport> <dst.dport> <seq>
<pktID> 
<type> := tcp|telnet|cbr|ack etc.
<size> := packet size in bytes
<flags> := [CP] C=congestion, P=priority 
<flowID> := flow identifier field as defined for IPv6 
<src.sport> := transport address (src=node,sport=agent) 
<dst.sport> := transport address (dst=node,dport=agent) 
<seq> := packet sequence number
<pktID> := unique identifer for every new packet 
\end{verbatim}

Only those agents interested in providing sequencing will generate
sequence numbers and hence this field may not be useful for packets
generated by some agents. For links that use RED gateways, there are
additional trace records as follows: \\

<code> <time> <value> \\

where \\
\begin{verbatim}
<code> := [Qap] Q=queue size, a=average queue size, p=packet dropping
probability
<time> := simulation time in seconds 
<value> := value 
\end{verbatim}

Trace records for link dynamics are of the form: \\

<code> <time> <state> <src> <dst> \\

where \\
\begin{verbatim}
<code> := [v]
<time> := simulation time in seconds 
<state> := [link-up | link-down]
<src> := first node address of link 
<dst> := second node address of link 
\end{verbatim} 
\end{flushleft}


\endinput
