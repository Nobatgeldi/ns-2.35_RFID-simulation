\chapter{DCCP Agents}
\label{sec:dccpAgents}

\section{DCCP Agents}
\label{sec:dccpagent}
This section describes the operation of the DCCP agents in \emph{ns}.
At current \emph{ns} implementation, there are two major congestion 
control of DCCP agents: CCID2 and CCID3.
It is a symmetric two-way agent in the sense that it represents
both a sender and receiver.
DCCP for \emph{ns} is still under development.

The files described in this section are too numerous to enumerate here.
Basically it covers most files matching the regular expression
\textasciitilde\emph{ns}/{dccp*.\{cc, h\}}.

Applications can access DCCP agents via the \fcn[]{sendmsg} function in C++,
or via the {\tt send} or {\tt sendmsg} methods in OTcl, as described in
section \ref{sec:systemcalls}.  

The following is a simple example of how a DCCP CCID2 agent may be used in a program.  
In the example, the CBR traffic generator is started at time 1.0, at which time
the generator begins to periodically call the DCCP agent \fcn[]{sendmsg}
function.

\begin{verbatim}
        set ns [new Simulator]
        set sender [\$ns node]
        set receiver [\$ns node]
        \$ns duplex-link \$sender \$receiver 5Mb 2ms DropTail

        set dccp0 [new Agent/DCCP/TCPlike]
        \$dccp0 set window\_ 7000
        set dccpsink0 [new Agent/DCCP/TCPlike]
        \$ns attach-agent \$sender \$dccp0
        \$ns attach-agent \$receiver \$dccpsink0
      
        set cbr0 [new Application/Traffic/CBR]
        \$cbr0 attach-agent \$dccp0
        \$cbr0 set packetSize\_ 160
        \$cbr0 set rate\_ 80Kb

        \$ns connect \$dccp0 \$dccpsink0
        \$ns at 1.0 "\$cbr0 start"
\end{verbatim}

The following example uses DCCP CCID3.
\begin{verbatim}
        set ns [new Simulator]
        set sender [\$ns node]
        set receiver [\$ns node]
        \$ns duplex-link \$sender \$receiver 5Mb 2ms DropTail

        set dccp0 [new Agent/DCCP/]
        set dccpsink0 [new Agent/DCCP/TFRC]
        \$ns attach-agent \$sender \$dccp0
        \$ns attach-agent \$receiver \$dccpsink0
      
        set cbr0 [new Application/Traffic/CBR]
        \$cbr0 attach-agent \$dccp0
        \$cbr0 set packetSize\_ 160
        \$cbr0 set rate\_ 80Kb

        \$ns connect \$dccp0 \$dccpsink0
        \$ns at 1.0 "\$cbr0 start"
\end{verbatim}

\section{Commands at a glance}
\label{sec:dccpcommand}

The following commands are used to setup DCDP agents in simulation scripts:
\begin{flushleft}
{\tt set dccp0 [new Agent/DCCP/TCPlike]}\\
This creates an instance of the DCCP CCID2 agent.

{\tt set dccp0 [new Agent/DCCP/TFRC]}\\
This creates an instance of the DCCP CCID3 agent.

{\tt \$ns\_ attach-agent \<node\> \<agent\>}\\
This is a common command used to attach any <agent> to a given <node>.

{\tt \$traffic-gen attach-agent \<agent\>}\\
This a class Application/Traffic/<traffictype> method which connects the
traffic generator to the given <agent>. For example, if we want to setup
a CBR traffic flow for the dccp agent, dccp0, we given the following commands\\
\begin{verbatim}
set cbr1 [new Application/Traffic/CBR]
\$cbr1 attach-agent \$dccp0
\end{verbatim}

For a more complex example of setting up an DCCP agent used in a simulation, see
the example code in tcl/ex folder.

\end{flushleft}

\endinput