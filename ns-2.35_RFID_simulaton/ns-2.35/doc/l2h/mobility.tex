\chapter{Mobile Networking in ns}
\label{chap:mobility}

This chapter describes the wireless model that was originally ported as CMU's Monarch group's mobility extension to \emph{ns}. 
This chapter consists of two sections and several subsections. The
first section covers the original mobility model ported from
CMU/Monarch group. In this section, we cover the internals of a
mobilenode, routing mechanisms and network components that are used to
construct the network stack for a mobilenode. The components that are
covered briefly are Channel, Network-interface, Radio propagation
model, MAC protocols, Interface Queue, Link layer and Address
resolution protocol model (ARP). CMU trace support and Generation of
node movement and traffic scenario files are also covered in this
section. 
The original CMU model allows simulation of pure wireless LANs or
multihop ad-hoc networks. Further extensions were made to this model
to allow combined simulation of wired and wireless networks. MobileIP
was also extended to the wireless model. These are
discussed in the second section of this chapter.                


\section{The basic wireless model in ns}
\label{sec:basic-model}

The wireless model essentially consists of the MobileNode at the core,with
additional supporting features that allows simulations of multi-hop ad-hoc
networks, wireless LANs etc. The MobileNode object is a split object. The
C++ \clsref{MobileNode}{../ns-2/mobilenode.h} is derived from parent
\clsref{Node}{../ns-2/node.h}. Refer to Chapter~\ref{chap:nodes} for
details on {\tt Node}. A {\tt MobileNode} thus is the basic {\tt Node}
object with added functionalities of a wireless and mobile node like
ability to move within a given topology, ability to receive and transmit
signals to and from a wireless channel etc. A major difference between
them, though, is that a {\tt MobileNode} is not connected by means of
{\tt Links} to other nodes or mobilenodes. In this section we shall
describe the internals of {\tt MobileNode}, its routing mechanisms, the
routing protocols dsdv, aodv, tora, puma and dsr, creation of network stack
allowing channel
access in {\tt MobileNode}, brief description of each stack component,
trace support and movement/traffic scenario generation for wireless
simulations. 


\subsection{Mobilenode: creating wireless topology}
\label{sec:mobilenode-creation}

{\tt MobileNode} is the basic \emph{ns} {\tt Node} object with added
functionalities like movement, ability to transmit and receive on a
channel that allows it to be used to create mobile, wireless simulation
environments. The class MobileNode is derived from the base class Node.
{\tt MobileNode} is a split object. The mobility features including node
movement, periodic position updates, maintaining topology boundary etc are
implemented in C++ while plumbing of network components within
{\tt MobileNode} itself (like classifiers, dmux , LL, Mac, Channel etc)
have been implemented in Otcl. The functions and procedures described in
this subsection can be found in \textasciitilde\emph{ns}/{mobilenode.\{cc,h\}},
\textasciitilde\emph{ns}/{tcl/lib/ns-mobilenode.tcl}, \textasciitilde\emph{ns}/{tcl/mobility/dsdv.tcl},
\textasciitilde\emph{ns}/{tcl/mobility/dsr.tcl}, \textasciitilde\emph{ns}/{tcl/mobility/tora.tcl}. Example scripts
can be found in
\textasciitilde\emph{ns}/{tcl/ex/wireless-test.tcl} and \textasciitilde\emph{ns}/{tcl/ex/wireless.tcl}. While the
first example uses a small topology of 3 nodes, the second example runs
over a topology of 50 nodes. These scripts can be run simply by typing

\begin{verbatim}
\$ns tcl/ex/wireless.tcl (or /wireless-test.tcl)
\end{verbatim} %\$

The five ad-hoc routing protocols that are currently supported are 
Destination Sequence Distance Vector (DSDV), Dynamic Source Routing
(DSR), Temporally ordered Routing Algorithm (TORA), Adhoc On-demand
Distance Vector (AODV) and Protocol for Unified Multicasting Through
Announcements (PUMA).
The primitive to create a mobilenode is described below. Please note 
that the old APIs for creating a mobilenode depended on which routing 
protocol was used, like 
\begin{verbatim} set mnode [\$opt(rp)-create-mobile-node \$id] \end{verbatim} 
where 
\begin{verbatim} \$opt(rp) \end{verbatim} 
defines "dsdv", "aodv", "tora", "dsr" or "puma" and id is the index for the mobilenode. But the old API's use is being deprecated and the new API is described as follows:.

\begin{verbatim}
\$ns\_ node-config -adhocRouting \$opt(adhocRouting) 
                 -llType \$opt(ll) 
                 -macType \$opt(mac) 
                 -ifqType \$opt(ifq) 
                 -ifqLen \$opt(ifqlen) 
                 -antType \$opt(ant) 
                 -propInstance [new \$opt(prop)] 
                 -phyType \$opt(netif) 
                 -channel [new \$opt(chan)] 
                 -topoInstance \$topo 
                 -wiredRouting OFF 
                 -agentTrace ON 
                 -routerTrace OFF 
                 -macTrace OFF
\end{verbatim} %\$

The above API configures for a mobilenode with all the given values of adhoc-routing protocol, network stack, channel,topography, propagation model, with wired routing turned on or off (required for wired-cum-wireless scenarios) and tracing turned on or off at different levels (router, mac, agent). Incase hierarchical addressing is being used, the hier address of the node needs to be passed as well. For more info about this command (part of new node APIs) see chapter titled "Restructuring ns node and new Node APIs" in ns Notes and Documentation.

Next actually create the mobilenodes as follows:
\begin{verbatim}
for { set j 0 } { \$j \< \$opt(nn)} {incr j} {
    set node\_(\$j) [ \$ns\_ node ]
    \$node\_(\$i) random-motion 0		;# disable random motion
}
\end{verbatim} %\$

The above procedure creates a mobilenode (split)object, creates an adhoc-routing routing agent as specified, creates the network stack consisting of a link layer, interface queue, mac layer, and a network interface with an antenna, uses the defined propagation model, interconnects these components and connects the stack to the channel. The mobilenode now looks like the schematic in Figure~\ref{fig:mobilenode-dsdv}.  
\begin{figure}
    \centerline{\includegraphics{dsdv}}
    \caption{Schematic of a mobilenode under the CMU monarch's
      wireless extensions to \emph{ns}} 
    \label{fig:mobilenode-dsdv} 
\end{figure}

The mobilenode structure used for DSR routing is slightly different from
the mobilenode described above. The class SRNode is derived from class
MobileNode. SRNode doesnot use address demux or classifiers and all
packets received by the node are handed dow   
n to the DSR routing agent by default. The DSR routing agent either
receives pkts for itself by handing it over to the port dmux or forwards
pkts as per source routes in the pkt hdr or sends out route requests and
route replies for fresh packets. Details    
on DSR routing agent may be found in section~\ref{sec:dsr}. The schematic
model for a SRNode is shown in Figure~\ref{fig:mobilenode-dsr}. 
\begin{figure}[tb]
    \centerline{\includegraphics{dsr}}
    \caption{Schematic of a SRNode under the CMU monarch's wireless
      extensions to \emph{ns}} 
    \label{fig:mobilenode-dsr}
\end{figure}

\subsection{Creating Node movements}
\label{sec:mobilenode-movements}

The mobilenode is designed to move in a three dimensional topology. However the third dimension (Z) is not used. That is the mobilenode is assumed to move always on a flat terrain with Z always equal to 0.
Thus the mobilenode has X, Y, Z(=0) co-ordinates that is continually adjusted as the node moves. There are two mechanisms to induce movement in mobilenodes. 
In the first method, starting position of the node and its future destinations may be set explicitly. These directives are normally included in a separate movement scenario file. 

The start-position and future destinations for a mobilenode may be set
by using the following APIs:
\begin{verbatim}
\$node set X\_ \<x1\>
\$node set Y\_ \<y1\>
\$node set Z\_ \<z1\>

\$ns at \$time \$node setdest \<x2\> \<y2\> \<speed\> 
\end{verbatim}
At \$time sec, the node would start moving from its initial position 
of (x1,y1) towards a destination (x2,y2) at the defined speed.

In this method the node-movement-updates are triggered whenever the
position of the node at a given time is required to be known. This
may be triggered by a query from a neighbouring node seeking to know
the distance between them, or the setdest directive
described above that changes the direction and speed of the node.

An example of a movement scenario file using the above APIs, can be
found in \textasciitilde\emph{ns}/{tcl/mobility/scene/scen-670x670-50-600-20-0}. Here
670x670 defines the length and width of the topology with 50 nodes
moving at a maximum speed of 20m/s with average pause time of
600s. These node movement files may be generated using CMU's scenario
generator to be found under
\textasciitilde\emph{ns}/{indep-utils/cmu-scen-gen/setdest}. See 
subsection~\ref{sec:mobile-scen-generator} for details on generation
of node movement scenarios. 

The second method employs random movement of the node. The primitive
to be used is:
\begin{verbatim}
\$mobilenode start
\end{verbatim} %\$
which starts the mobilenode with a random position and have routined
updates to change the direction and speed of the node. The destination
and speed values are generated in a random fashion. We have not used
the second method and leave it to the user to 
explore the details. 
The mobilenode movement is implemented in C++. See methods in
\textasciitilde\emph{ns}/{mobilenode.\{cc.h\}} for the implementational details.

Irrespective of the methods used to generate node movement,
the topography for mobilenodes needs to be defined. It should be
defined before creating mobilenodes. Normally flat topology is created
by specifying the length and width of the topography using the
following primitive:
\begin{verbatim}    
set topo        [new Topography]
\$topo load\_flatgrid \$opt(x) \$opt(y)
\end{verbatim} %\$
where opt(x) and opt(y) are the boundaries used in simulation.

The movement of mobilenodes may be logged by using a procedure like
the following:

\begin{verbatim}
proc log-movement {} {
    global logtimer ns\_ ns

    set ns \$ns\_
    source ../mobility/timer.tcl
    Class LogTimer -superclass Timer
    LogTimer instproc timeout {} {
        global opt node\_;
        for {set i 0} {\$i \< \$opt(nn)} {incr i} {
            \$node\_(\$i) log-movement
        }
        \$self sched 0.1
    }

    set logtimer [new LogTimer]
    \$logtimer sched 0.1
}
\end{verbatim} %\$
In this case, mobilenode positions would be logged every 0.1 sec.

\subsection{Network Components in a mobilenode}
\label{sec:mobilenode-components}

The network stack for a mobilenode consists of a link layer(LL), an
ARP module connected to LL, an interface priority queue(IFq), a mac
layer(MAC), a network interface(netIF), all connected to the channel. 
These network components are created and plumbed together in OTcl. 
The relevant MobileNode method add-interface() in
\textasciitilde\emph{ns}/{tcl/lib/ns-mobilenode.tcl} is shown below:

\begin{verbatim}
#
#  The following setups up link layer, mac layer, network interface
#  and physical layer structures for the mobile node.
#
Node/MobileNode instproc add-interface { channel pmodel 
                lltype mactype qtype qlen iftype anttype } {

        \$self instvar arptable\_ nifs\_
        \$self instvar netif\_ mac\_ ifq\_ ll\_

        global ns\_ MacTrace opt

        set t \$nifs\_
        incr nifs\_

        set netif\_(\$t)  [new \$iftype]           ;# net-interface
        set mac\_(\$t)    [new \$mactype]          ;# mac layer
        set ifq\_(\$t)    [new \$qtype]            ;# interface queue
        set ll\_(\$t)     [new \$lltype]           ;# link layer
        set ant\_(\$t)    [new \$anttype]

        #
        # Local Variables
        #
        set nullAgent\_ [\$ns\_ set nullAgent\_]
        set netif \$netif\_(\$t)
        set mac \$mac\_(\$t)
        set ifq \$ifq\_(\$t)
        set ll \$ll\_(\$t)

        #
        # Initialize ARP table only once.
        #
        if { \$arptable\_ == "" } {
            set arptable\_ [new ARPTable \$self \$mac]
            set drpT [cmu-trace Drop "IFQ" \$self]
            \$arptable\_ drop-target \$drpT
        }

        #
        # Link Layer
        #
        \$ll arptable \$arptable\_
        \$ll mac \$mac
        \$ll up-target [\$self entry]
        \$ll down-target \$ifq

        #
        # Interface Queue
        #
        \$ifq target \$mac
        \$ifq set qlim\_ \$qlen
        set drpT [cmu-trace Drop "IFQ" \$self]
        \$ifq drop-target \$drpT

        #
        # Mac Layer
        #
        \$mac netif \$netif
        \$mac up-target \$ll
        \$mac down-target \$netif
        \$mac nodes \$opt(nn)

        #
        # Network Interface
        #
        \$netif channel \$channel
        \$netif up-target \$mac
        \$netif propagation \$pmodel      ;# Propagation Model
        \$netif node \$self               ;# Bind node \<---\> interface
        \$netif antenna \$ant\_(\$t)        ;# attach antenna

        #
        # Physical Channel
        #
        \$channel addif \$netif           ;# add to list of interfaces

        # ============================================================
        # Setting up trace objects
        
        if { \$MacTrace == "ON" } {
            #
            # Trace RTS/CTS/ACK Packets
            #
            set rcvT [cmu-trace Recv "MAC" \$self]
            \$mac log-target \$rcvT


            #
            # Trace Sent Packets
            #
            set sndT [cmu-trace Send "MAC" \$self]
            \$sndT target [\$mac sendtarget]
            \$mac sendtarget \$sndT

            #
            # Trace Received Packets
            #
            set rcvT [cmu-trace Recv "MAC" \$self]
            \$rcvT target [\$mac recvtarget]
            \$mac recvtarget \$rcvT

            #
            # Trace Dropped Packets
            #
            set drpT [cmu-trace Drop "MAC" \$self]
            \$mac drop-target \$drpT
        } else {
            \$mac log-target [\$ns\_ set nullAgent\_]
            \$mac drop-target [\$ns\_ set nullAgent\_]
        }

        # ============================================================

        \$self addif \$netif
}
\end{verbatim} %\$

The plumbing in the above method creates the network stack we see in
Figure~\ref{fig:mobilenode-dsdv}.

Each component is briefly described here. Hopefully more detailed
docuentation from CMU shall be available in the future. 
\begin{description}
\item[{\bf Link Layer}] The {\tt LL} used by mobilenode is same as
  described in Chapter~\ref{chap:lan}. The only difference being the
  link layer for mobilenode, has an ARP module connected to it which
  resolves all IP to hardware (Mac) address conversions. Normally for
  all outgoing (into the channel) packets, the packets are handed down
  to the {\tt LL} by the Routing Agent. The {\tt LL} hands down
  packets to the interface queue. For all incoming packets (out of the
  channel), the mac layer hands up packets to the {\tt LL} which is
  then handed off at the {\tt node\_entry\_} point. The
  \clsref{LL}{../ns-2/ll.h} is implemented in \textasciitilde\emph{ns}/{ll.\{cc,h\}} and
  \textasciitilde\emph{ns}/{tcl/lan/ns-ll.tcl}.

\item [{\bf ARP}] The Address Resolution Protocol (implemented in BSD
  style) module receives queries from Link layer. If ARP has the
  hardware address for destination, it writes it into the mac header
  of the packet. Otherwise it broadcasts an ARP query, and caches the
  packet temporarily. For each unknown destination hardware address,
  there is a buffer for a single packet. Incase additional packets to
  the same destination is sent to ARP, the earlier buffered packet is
  dropped. Once the hardware address of a
  packet's next hop is known, the packet is inserted into the
  interface queue. The \clsref{ARPTable}{../ns-2/arp.h} is implemented
  in \textasciitilde\emph{ns}/{arp.\{cc,h\}} and \textasciitilde\emph{ns}/{tcl/lib/ns-mobilenode.tcl}.

\item[{\bf Interface Queue}] The \clsref{PriQueue}{../ns-2/priqueue.h}
  is implemented as a priority queue which gives priority to routing  
  rotocol packets, inserting them at the head of the queue. It supports
  running a filter over all packets in the queue and removes those with
  a specified destination address. See \textasciitilde\emph{ns}/{priqueue.\{cc,h\}} for 
  interface queue implementation.

\item[{\bf Mac Layer}] Historically, ns-2 (prior to release ns-2.33)
  has used the implementation of IEEE 802.11 distributed coordination 
  function (DCF) from CMU. Starting with ns-2.33, several 802.11 
  implementations are available.  See section \ref{sec:802_11} for
  more information.

\item[{\bf Tap Agents}] {\tt Agents} that subclass themselves as
  \clsref{Tap}{../ns-2/mac.h} defined in mac.h can register themselves
  with the mac object using method installTap(). If the particular Mac
  protocol permits it, the tap will promiscuously be 
  given all packets received by the mac layer, before address filtering
  is done. See \textasciitilde\emph{ns}/{mac.\{cc,h\}} for \clsref{Tap} implementation. 

\item[{\bf Network Interfaces}] The Network Interphase layer serves as
  a hardware interface which is used by mobilenode to access the
  channel. The wireless shared media interface is implemented as
  \clsref{Phy/WirelessPhy}{../ns-2/wireless-phy.h}. This interface
  subject to collisions and the radio propagation model receives
  packets transmitted by other node interfaces to the channel. The
  interface stamps each transmitted packet with the meta-data related
  to the transmitting interface like the transmission power,
  wavelength etc. This meta-data in pkt header is used by the
  propagation model in receiving network interface to determine if the
  packet has minimum power to be received and/or captured and/or
  detected (carrier sense) by the receiving node. The model
  approximates the DSSS radio interface (Lucent WaveLan
  direct-sequence spread-spectrum). See \textasciitilde\emph{ns}/{phy.\{cc.h\}} and
  \textasciitilde\emph{ns}/{wireless-phy.\{cc,h\}} for network interface implementations.

\item[{\bf Radio Propagation Model}]  It uses Friss-space attenuation
  ($1/r^2$) at near distances and an approximation to Two ray Ground
  ($1/r^4$) at far distances. The approximation assumes specular
  reflection off a flat ground plane. See \textasciitilde\emph{ns}/{tworayground.\{cc,h\}}
  for implementation.

\item[{\bf Antenna}] An omni-directional antenna having unity gain is 
  used by mobilenodes. See \textasciitilde\emph{ns}/{antenna.\{cc,h\}} for implementation
  details. 
\end{description}

\subsection{Different MAC layer protocols for mobile networking}
\label{sec:mobilenode-mac}

In \emph{ns}, two MAC layer protocols are implemented for mobile networks,
  which are 802.11 and TDMA.
In this section
  we briefly discuss each of them.

\subsubsection{802.11 MAC protocol}
\label{sec:802_11_brief}

  Historically, ns-2 (prior to release ns-2.33)
  has used the implementation of IEEE 802.11 distributed coordination 
  function (DCF) from CMU. Starting with ns-2.33, several 802.11 
  implementations are available.  See section \ref{sec:802_11} for
  more information.

\subsubsection{Preamble based TDMA protocol}
\label{sec:tdma}
{\bf Note:} this works is still at a preliminary stage,
  some practical issues, such as:
  contention in the preamble phase and 
  time slot reuse in a multi-hop environment are not considered.

Unlike contention based MAC protocol (802.11, for example), 
  a TDMA MAC protocol allocates different time slots for nodes to 
  send and receive packets.
The superset of these time slots is called a TDMA frame.

Currently, \emph{ns}\ supports a single hop, 
  preamble-based TDMA MAC protocol.
With this protocl,
  a TDMA frame contains preamble besides the data transmission slots.
Within the preamble,
  every node has a dedicated subslot and
  uses it to broadcast the destination node id of outgoing packet.
Other nodes listen in the preamble and record 
  the time slots to receive packets.
Like other common TDMA protocols (GSM, for example),
  each node has a data transmission slot to send packets.

To avoid unnecessary power consumption,
  each node turns its radio on and off explicitly
  by invoking node API {\tt set\_node\_sleep()}.
The radio only needs to be on when:
  in the pramble phase (takes one slot time) and
  there is a packet to send and receive.

The preamble is implemented as a central data structure
  {\tt tdma\_preamble\_},
  which is accessible to all the nodes.
At the beginning of a frame,
  each node writes the destination node id into
  its subslot in preamble if it has a packet to send.
Following preamble phase,
  each node sends packet in its data transmission slot and
  checks the preamble to determine if 
  there is a packet to receive in other slots.

The following parameters are user configurable:
 the wireless link bandwidth {\tt bandwith\_},  
  the slot length {\tt packet\_slot\_len\_}, 
  and the number of nodes {\tt max\_node\_num\_}.
See \textasciitilde\emph{ns}/{mac-tdma.\{cc,h\}} for implementation details.

\subsection{Different types of Routing Agents in mobile networking}
\label{sec:mobilenode-routing}

The five different ad-hoc routing protocols currently implemented
for mobile networking in \emph{ns} are dsdv, dsr, aodv, tora and puma. 
In this section we shall briefly discuss each of them.

\subsubsection{DSDV}
\label{sec:dsdv}

In this routing protocol routing messages are exchanged between
neighbouring mobilenodes (i.e mobilenodes that are within range of one
another). Routing updates may be triggered or routine. Updates are
triggered in case a routing information from one of t   
he neighbours forces a change in the routing table.
A packet for which the route to its destination is not known is cached
while routing queries are sent out. The pkts are cached until
route-replies are received from the destination. There is a maximum buffer
size for caching the pkts waiting for routing information beyond which
pkts are dropped. 

All packets destined for the mobilenode are routed directly by the address
dmux to its port dmux. The port dmux hands the packets to the respective
destination agents. A port number of 255 is used to attach routing agent
in mobilenodes. The mobilenodes al
so use a default-target in their classifier (or address demux). In the
event a target is not found for the destination in the classifier (which
happens when the destination of the packet is not the mobilenode itself),
the pkts are handed to the default-ta   
rget which is the routing agent. The routing agent assigns the next hop
for the packet and sends it down to the link layer. 

The routing protocol is mainly implemented in C++. See \textasciitilde\emph{ns}/{dsdv}
directory and \textasciitilde\emph{ns}/{tcl/mobility/dsdv.tc}l for all procedures related to
DSDV protocol implementation.  

\subsubsection{DSR}
\label{sec:dsr}

This section briefly describes the functionality of the dynamic source
routing protocol. As mentioned earlier the {\tt SRNode} is different from
the {\tt MobileNode}.  The {\tt SRNode}'s {\tt entry\\_} points to the
DSR routing agent, thus forcing all    
packets received by the node to be handed down to the routing agent. This
model is required for future implementation of piggy-backed routing
information on data packets which otherwise would not flow through the
routing agent.   

The DSR agent checks every data packet for source-route information. It
forwards the packet as per the routing information. Incase it doesnot find
routing information in the packet, it provides the source route, if route
is known, or caches the packet and   
sends out route queries if route to destination is not known. Routing
queries, always triggered by a data packet with no route to its
destination, are initially broadcast to all neighbours. Route-replies are
send back either by intermediate nodes or the 
destination node, to the source, if it can find routing info for the
destination in the route-query.  It hands over all packets destined to
itself to the port dmux.  
In {\tt SRNode} the port number 255 points to a null agent since the
packet has already been processed by the routing agent. 

See \textasciitilde\emph{ns}/{dsr} directory and \textasciitilde\emph{ns}/{tcl/mobility/dsr.tcl} for implementation
of DSR protocol. 

\subsubsection{TORA}
\label{sec:tora}

Tora is a distributed routing protocol based on "link reversal" algorithm. 
At every node a separate copy of TORA is run for every destination. When a
node needs a route to a given destination it broadcasts a QUERY message
containing the address of the destination for which it requires  a route.
This packet travels through the network until it reaches the destination
or an intermediate node that has a route to the destination node.
This recepient node node then broadcasts an UPDATE packet listing its
height wrt the destination. As this node propagates through the network
each node updates its height to a value greater than the height of the
neighbour from which it receives the UPDATE. This results in a series of
directed links from the node that originated the QUERY to the destination
node. If a node discovers a particular destination to be unreachable it
sets a local maximum value of height for that destination. Incase the node
cannot find any neighbour having finite height wrt this destination it
attempts to find a new route. In case of network partition, the node
broadcasts a CLEAR message that resets all routing states and removes
invalid routes from the network.

TORA operates on top of IMEP (Internet MANET Encapsulation Protocol) that
provides reliable delivery of route-messages and informs the routing
protocol of any changes of the links to its neighbours. IMEP tries to
aggregate IMEP and TORA messages into a single packet (called block) in
order to reduce overhead. For link-status sensing and maintaining a list
of neighbour nodes, IMEP sends out periodic BEACON messages which is
answered by each node that hears it by a HELLO reply message.
See \emph{ns}/tora directory and \emph{ns}/tcl/mobility/tora.tcl for implementation of
tora in \emph{ns}.

\subsubsection{AODV}
\label{sec:AODV}

AODV is a combination of both DSR and DSDV protocols. It has the basic
route-discovery and route-maintenance of DSR and uses the hop-by-hop
routing, sequence numbers and beacons of DSDV. The node that wants to know
a route to a given destination generates a ROUTE REQUEST. The route
request is forwarded by intermediate nodes that also creates a reverse
route for itself from the destination. When the request reaches a node
with route to destination it generates a ROUTE REPLY containing the number
of hops requires to reach destination. All nodes that participates in
forwarding this reply to the source node creates a forward route to
destination. This state created from each node from source to destination
is a hop-by-hop state and not the entire route as is done in source
routing.
See \emph{ns}/aodv and \emph{ns}/tcl/lib/ns-lib.tcl for implementational details
of aodv.

\subsubsection{PUMA}
\label{sec:PUMA}

The \emph{Protocol for Unified Multicasting Through Announcements} (PUMA) 
is a distributed, receiver initiated, mesh based multicast routing protocol. 
By default, the first receiver in a multicast group acts as the core (i.e., 
\emph{rendezvous point}) for that particular group. PUMA uses a simple and 
very efficient control message, a \textit{multicast announcement}, to 
maintain the mesh. Besides that, multiple meshes can be compiled into a 
single announcement bucket. PUMA does not require any unicast protocol, and
all transmissions are broadcasts. Even though broadcast transmissions are 
unreliable, the mesh itself introduces some redundancy, and because the mesh
includes only group members and the nodes interconnecting them, broadcasts
remain scoped within the mesh.

As a multicast announcement propagates throughout the mesh, nodes learn the
shortest path to the core. This way, data packets can be quickly routed to 
the core. On its way toward the core, two things can happen to a data packet:
(a) the packet goes all the way until it reaches the core, or (b) a mesh 
member is hit before reaching the core. Anyway, once a data packet reaches 
the mesh, the packet propagates only inside the mesh. The core is not a 
single point of failure, because when the core fails a group member quickly 
takes the core role.
See \emph{ns}/puma directory and \emph{ns}/tcl/ex/puma.tcl for implementation of
PUMA in \emph{ns}.

\subsubsection{M-DART}
\label{sec:M-DART}

The Multi-Path Dynamic Addressing Routing (M-DART) is a routing protocol 
for ad hoc networks with the following features:
\begin{itemize}
\item proactive, every node keeps information about the available routes;
\item multi-path, every node tracks redundant routes to face with topology changes;
\item hierarchic, the routing overhead is reduced with a logarithmic factor;
\item DHT-based, since a DHT is used at the network layer.
\end{itemize}

The M-DART extends the DART protocol, first proposed by 
J. Eriksson, M. Faloutsos and S. Krishnamurthy.  The ns-2 
implementation has been extensively tested for ad hoc networks up to 
4096 nodes. The ./mdart/example/ folder contains an example of static 
ad hoc network scenario used for the results published in 
M. Caleffi, L. Paura, "On Reliability of Dynamic Addressing Routing 
Protocols in Mobile Ad Hoc Networks", accepted for publication in 
the special issue on "Architectures and Protocols for Wireless Mesh, Ad Hoc,
and Sensor Networks" of Wireless Communications and Mobile Computing, 2010.

For more information:
\begin{itemize}
\item M. Caleffi, G. Ferraiuolo, L. Paura, "Augmented Tree-based Routing  
Protocol for Scalable Ad Hoc Networks", Proc. of IEEE MASS '07: IEEE      *
Internatonal Conference on Mobile Adhoc and Sensor Systems, Pisa (Italy), *
October 8-11 2007. 
\item M. Caleffi, "Mobile Ad Hoc Networks: the DHT Paradigm, Ph.D. Thesis", 
University of Naples Federico II, December 2008. 
\end{itemize}

\subsection{Trace Support}
\label{sec:mobile-trace}

The trace support for wireless simulations currently use cmu-trace
objects. In the future this shall be extended to merge with trace and
monitoring support available in ns, which would also include nam support
for wireless modules. For now we will explain briefly with cmu-trace
objects and how they may be used to trace packets for wireless scenarios. 

The cmu-trace objects are of three types - {\tt CMUTrace/Drop},
{\tt CMUTrace/Recv} and {\tt CMUTrace/Send}. These are used for tracing
packets that are dropped, received and sent by agents, routers, mac layers
or interface queues in \emph{ns}. The methods and procedures used for
implementing wireless trace support can be found under
\textasciitilde\emph{ns}/{trace.\{cc,h\}} and \textasciitilde\emph{ns}/{tcl/lib/ns-cmutrace.tcl}.

A cmu-trace object may be created by the following API:
\begin{verbatim}
set sndT [cmu-trace Send "RTR" \$self]
\end{verbatim} %\$
which creates a trace object, sndT, of the type {\tt CMUTrace/Send}
for tracing all packets that are sent out in a router. The trace
objects may be used to trace packets in MAC, agents (routing or
others), routers or any other NsObject. 

The cmu-trace object {\tt CMUTrace} is derived from the base class
{\tt Trace}. See Chapter~\ref{chap:trace} for details on class
{\tt Trace}. The class {\tt CMUTrace} is defined as the following:

\begin{verbatim}
class CMUTrace : public Trace {
public:
        CMUTrace(const char *s, char t);
        void    recv(Packet *p, Handler *h);
        void    recv(Packet *p, const char* why);

private:
        int off\_arp\_;
        int off\_mac\_;
        int off\_sr\_;

        char    tracename[MAX\_ID\_LEN + 1];
        int     tracetype;
        MobileNode *node\_;

        int initialized() { return node\_ && 1; }

        int     command(int argc, const char*const* argv);
        void    format(Packet *p, const char *why);

        void    format\_mac(Packet *p, const char *why, int offset);
        void    format\_ip(Packet *p, int offset);

        void    format\_arp(Packet *p, int offset);
        void    format\_dsr(Packet *p, int offset);
        void    format\_msg(Packet *p, int offset);
        void    format\_tcp(Packet *p, int offset);
        void    format\_rtp(Packet *p, int offset);
};
\end{verbatim}

The type field (described in {\tt Trace} class definition) is used to
differentiate among different types of traces. For cmu-trace this can be
{\bf s} for sending, {\bf r} for receiving or {\bf D} for dropping a
packet. A fourth type {\bf f} is used to denote forwarding of a packet
(When the node is not the originator of the packet). 
Similar to the method Trace::format(), the CMUTrace::format() defines and
dictates the trace file format. The method is shown below: 
\begin{verbatim}
void CMUTrace::format(Packet* p, const char *why)
{
        hdr\_cmn *ch = HDR\_CMN(p);
        int offset = 0;

        /*
         * Log the MAC Header
         */
        format\_mac(p, why, offset);
        offset = strlen(wrk\_);

        switch(ch-\>ptype()) {

        case PT\_MAC:
                break;

        case PT\_ARP:
                format\_arp(p, offset);
                break;

        default:
                format\_ip(p, offset);
                offset = strlen(wrk\_);

                switch(ch-\>ptype()) {

                case PT\_DSR:
                        format\_dsr(p, offset);
                        break;

                case PT\_MESSAGE:
                case PT\_UDP:
                        format\_msg(p, offset);
                        break;
                        
                case PT\_TCP:
                case PT\_ACK:
                        format\_tcp(p, offset);
                        break;
                        
                case PT\_CBR:
                        format\_rtp(p, offset);
                        break;
                ..........

                }
        }
}
\end{verbatim}
The above function calls different format functions depending on the type
of the packet being traced. All traces are written to the buffer wrk\_. A
count of the offset for the buffer is kept and is passed along the
different trace functions. The most basic format is defined by
format\_mac() and is used to trace all pkt types. The other format
functions print additional information as defined by the packet types. The
mac format prints the following:   
\begin{verbatim}
\#ifdef LOG\_POSITION
        double x = 0.0, y = 0.0, z = 0.0;
        node\_-\>getLoc(&x, &y, &z);
\#endif

        sprintf(wrk\_ + offset,
\#ifdef LOG\_POSITION
                "%c %.9f %d (%6.2f %6.2f) %3s %4s %d %s %d [%x %x %x %x] ",
\#else
                "%c %.9f \_%d\_ %3s %4s %d %s %d [%x %x %x %x] ",
\#endif
                op,                    // s, r, D or f
                Scheduler::instance().clock(),  // time stamp
                src\_,                  // the nodeid for this node
\#ifdef LOG\_POSITION
                x,                     // x co-ord 
                y,                     // y co-ord
\#endif
                tracename,             // name of object type tracing
                why,                   // reason, if any

                ch-\>uid(),             // identifier for this event
                packet\_info.name(ch-\>ptype()), // packet type
                ch-\>size(),                    // size of cmn header
                mh-\>dh\_duration,       // expected time to send data 
                ETHER\_ADDR(mh-\>dh\_da), // mac\_destination address
                ETHER\_ADDR(mh-\>dh\_sa),         // mac\_sender address
                GET\_ETHER\_TYPE(mh-\>dh\_body));  // type - arp or IP
\end{verbatim}

If the LOG\_POSITION is defined the x and y co-ordinates for the
mobilenode is also printed. The descriptions for different fields in the
mac trace are given in the comments above. For all IP packets additional
IP header fields are also added to the above trace. The IP trace is
described below:

\begin{verbatim}
sprintf(wrk\_ + offset, "------- [%d:%d %d:%d %d %d] ",
                src,          // IP src address
                ih-\>sport\_,   // src port number
                dst,          // IP dest address
                ih-\>dport\_,   // dest port number
                ih-\>ttl\_,     // TTL value 
                (ch-\>next\_hop\_ \< 0) ? 0 : ch-\>next\_hop\_); // next hopaddress, if any.
\end{verbatim}

An example of a trace for a tcp packet is as follows:
\begin{verbatim}
r 160.093884945 \_6\_ RTR  --- 5 tcp 1492 [a2 4 6 800] ------- [655
36:0 16777984:0 31 16777984] [1 0] 2 0
\end{verbatim}
Here we see a TCP data packet being received by a node with id of 6. UID
of this pkt is 5 with a cmn hdr size of 1492. The mac details shows an IP
pkt (ETHERTYPE\_IP is defined as 0x0800, ETHERTYPE\_ARP is 0x0806 ), mac-id
of this receiving node is 4. That of the sending node is 6 and expected
time to send this data pkt over the wireless channel is a2 (hex2dec
conversion: 160+2 sec). Additionally, IP traces information about IP src
and destination addresses. The src translates (using a 3 level
hier-address of 8/8/8) to a address string of 0.1.0 with port of 0. The
dest address is 1.0.3 with port address of 0. The TTL value is 31 and the
destination was a hop away from the src. Additionally TCP format prints
information about tcp seqno of 1, ackno of 0. See other formats described
in \textasciitilde\emph{ns}//cmu-trace.cc for DSR, UDP/MESSAGE, TCP/ACK and CBR packet types.

Other trace formats are also used by the routing agents (TORA and DSR) to
log certain special routing events like "originating" (adding a SR header
to a packet) or  "ran off the end of a source route" indicating some sort
of routing problem with the source route etc. These special event traces
begin with "S" for DSR and "T" for Tora and may
be found in \textasciitilde\emph{ns}/{tora/tora.cc} for TORA and \textasciitilde\emph{ns}/{dsr/dsrgent.cc} for DSR
routing agent.


\subsection{Revised format for wireless traces}
\label{sec:revtraceformat}

In an effort to merge wireless trace, using cmu-trace objects, with
ns tracing, a new, inproved trace format has been introduced. This revised
trace support is backwards compatible with the old trace formatting and
can be enabled by the following command:
\begin{verbatim}
\$ns use-newtrace
\end{verbatim}
This command should be called before the universal trace command
{\tt \$ns trace-all \<trace-fd\>}. Primitive {\tt use-newtrace} sets up new
format for wireless tracing by setting a simulator variable called
{\tt newTraceFormat}. Currently this new trace support is available for
wireless simulations only and shall be extended to rest of \emph{ns}\ in the near
future.

An example of the new trace format is shown below:
\begin{verbatim}
s -t 0.267662078 -Hs 0 -Hd -1 -Ni 0 -Nx 5.00 -Ny 2.00 -Nz 0.00 -Ne
-1.000000 -Nl RTR -Nw --- -Ma 0 -Md 0 -Ms 0 -Mt 0 -Is 0.255 -Id -1.255 -It
message -Il 32 -If 0 -Ii 0 -Iv 32
s -t 1.511681090 -Hs 1 -Hd -1 -Ni 1 -Nx 390.00 -Ny 385.00 -Nz 0.00 -Ne
-1.000000 -Nl RTR -Nw --- -Ma 0 -Md 0 -Ms 0 -Mt 0 -Is 1.255 -Id -1.255 -It
message -Il 32 -If 0 -Ii 1 -Iv 32
s -t 10.000000000 -Hs 0 -Hd -2 -Ni 0 -Nx 5.00 -Ny 2.00 -Nz 0.00 -Ne
-1.000000 -Nl AGT -Nw --- -Ma 0 -Md 0 -Ms 0 -Mt 0 -Is 0.0 -Id 1.0 -It tcp -Il 1000 -If
2 -Ii 2 -Iv 32 -Pn tcp -Ps 0 -Pa 0 -Pf 0 -Po 0
r -t 10.000000000 -Hs 0 -Hd -2 -Ni 0 -Nx 5.00 -Ny 2.00 -Nz 0.00 -Ne
-1.000000 -Nl RTR -Nw --- -Ma 0 -Md 0 -Ms 0 -Mt 0 -Is 0.0 -Id 1.0 -It tcp -Il 1000 -If
2 -Ii 2 -Iv 32 -Pn tcp -Ps 0 -Pa 0 -Pf 0 -Po 0
r -t 100.004776054 -Hs 1 -Hd 1 -Ni 1 -Nx 25.05 -Ny 20.05 -Nz 0.00 -Ne
-1.000000 -Nl AGT -Nw --- -Ma a2 -Md 1 -Ms 0 -Mt 800 -Is 0.0 -Id 1.0 -It
tcp -Il 1020 -If 2 -Ii 21 -Iv 32 -Pn tcp -Ps 0 -Pa 0 -Pf 1 -Po 0
s -t 100.004776054 -Hs 1 -Hd -2 -Ni 1 -Nx 25.05 -Ny 20.05 -Nz 0.00 -Ne
-1.000000 -Nl AGT -Nw --- -Ma 0 -Md 0 -Ms 0 -Mt 0 -Is 1.0 -Id 0.0 -It ack -Il 40
-If 2 -Ii 22 -Iv 32 -Pn tcp -Ps 0 -Pa 0 -Pf 0 -Po 0 
\end{verbatim}

\subsubsection{Explanation of new trace format}
The new trace format as seen above can be can be divided into the
following fields :
\begin{description}
\item[Event type]
In the traces above, the first field (as in the older trace format)
describes the type of event taking place at the node and can be one of the 
four types:
\begin{description}
\item[s] send 
\item[r] receive 
\item[d] drop 
\item[f] forward
\end{description}

\item[General tag]
The second field starting with "-t" may stand for time or global setting
\begin{description}
\item[-t] time
\item[-t] * (global setting)
\end{description}

\item[Node property tags]
This field denotes the node properties like node-id, the level at
which tracing is being done like agent, router or MAC. The tags start
with a leading "-N" and are listed as below:
\begin{description}
\item[-Ni:] node id
\item[-Nx:] node's x-coordinate
\item[-Ny:] node's y-coordinate
\item[-Nz:] node's z-coordinate
\item[-Ne:] node energy level
\item[-Nl:] trace level, such as AGT, RTR, MAC
\item[-Nw:] reason for the event. The different reasons for dropping a
packet are given below:
\begin{description}
\item["END"] DROP\_END\_OF\_SIMULATION          
\item["COL"] DROP\_MAC\_COLLISION              
\item["DUP"] DROP\_MAC\_DUPLICATE 
\item["ERR"] DROP\_MAC\_PACKET\_ERROR  
\item["RET"] DROP\_MAC\_RETRY\_COUNT\_EXCEEDED
\item["STA"] DROP\_MAC\_INVALID\_STATE      
\item["BSY"] DROP\_MAC\_BUSY                 

\item["NRTE"] DROP\_RTR\_NO\_ROUTE i.e no route is available.
\item["LOOP"] DROP\_RTR\_ROUTE\_LOOP i.e there is a routing loop
\item["TTL"]  DROP\_RTR\_TTL i.e TTL has reached zero.
\item["TOUT"] DROP\_RTR\_QTIMEOUT i.e packet has expired.
\item["CBK"]  DROP\_RTR\_MAC\_CALLBACK

\item["IFQ"]  DROP\_IFQ\_QFULL i.e no buffer space in IFQ.
\item["ARP"]  DROP\_IFQ\_ARP\_FULL i.e dropped by ARP
\item["OUT"]  DROP\_OUTSIDE\_SUBNET i.e dropped by base stations on
receiving routing updates from nodes outside its domain.
\end{description}
\end{description}

\item[Packet information at IP level]
The tags for this field start with a leading "-I" and are listed along
with their explanations as following:
\begin{description}
\item[-Is:] source address.source port number
\item[-Id:] dest address.dest port number
\item[-It:] packet type
\item[-Il:] packet size
\item[-If:] flow id
\item[-Ii:] unique id
\item[-Iv:] ttl value  
\end{description}

\item[Next hop info]
This field provides next hop info and the tag starts with a leading "-H".
\begin{description}
\item[-Hs:] id for this node
\item[-Hd:] id for next hop towards the destination.
\end{description}

\item[Packet info at MAC level]
This field gives MAC layer information and starts with a leading "-M" as
shown below:
\begin{description}
\item[-Ma:] duration
\item[-Md:] dst's ethernet address
\item[-Ms:] src's ethernet address
\item[-Mt:] ethernet type  
\end{description}     

\item[Packet info at "Application level"]
The packet information at application level consists of the type of
application like ARP, TCP, the type of adhoc routing protocol like
PUMA, DSR, AODV etc being traced. This field consists of a leading "-P"
and list of tags for different application is listed as below:
\begin{description}
\item[-P arp] Address Resolution Protocol. Details for ARP is given by the
following tags:
\begin{description}
\item[-Po:] ARP Request/Reply
\item[-Pm:] src mac address
\item[-Ps:] src address
\item[-Pa:] dst mac address
\item[-Pd:] dst address
\end{description}

\item[-P dsr] This denotes the adhoc routing protocol called Dynamic
source routing. Information on DSR is represented by the following tags:
\begin{description}
\item[-Pn:] how many nodes traversed
\item[-Pq:] routing request flag
\item[-Pi:] route request sequence number
\item[-Pp:] routing reply flag
\item[-Pl:] reply length
\item[-Pe:] src of srcrouting->dst of the source routing
\item[-Pw:] error report flag ?
\item[-Pm:] number of errors
\item[-Pc:] report to whom
\item[-Pb:] link error from linka->linkb
\end{description}

\item[-P cbr] Constant bit rate. Information about the CBR application is
represented by the following tags:

\begin{description}
\item[-Pi:] sequence number
\item[-Pf:] how many times this pkt was forwarded
\item[-Po:] optimal number of forwards 
\end{description}

\item[-P tcp] Information about TCP flow is given by the following
subtags:

\begin{description}
\item[-Ps:] seq number
\item[-Pa:] ack number
\item[-Pf:] how many times this pkt was forwarded
\item[-Po:] optimal number of forwards 
\end{description}
\end{description}

This field is still under development and new tags shall be added for
other applications as they get included along the way.
\end{description}


\subsection{Generation of node-movement and traffic-connection for
  wireless scenarios}
\label{sec:mobile-scen-generator}

Normally for large topologies, the node movement and traffic connection
patterns are defined in separate files for convinience. These movement and
traffic files may be generated using CMU's movement- and
connection-generators. In this section we shall describe both separately.

\subsubsection{MobileNode Movement}
\label{sec:mobile-movement-file}

Some examples of node movement files may be found in
\textasciitilde\emph{ns}/{tcl/mobility/scene/scen-670x670-50-600-20-*}. These files
define a topology of 670 by 670m where 50 nodes move with a speed of 20m/s
with pause time of 600s. each node is assigned a starting position. The
information regarding number of hops between the nodes is fed to the
central object "GOD" (XXX but why/where is this information used??-answer
awaited from CMU.) Next each node is a speed and a direction to move to. 

The generator for creating node movement files are to be found under
\textasciitilde\emph{ns}/{indep-utils/cmu-scen-gen/setdest/} directory. Compile the files
under setdest to create an executable. run setdest with arguments in
the following way:
\begin{verbatim}
./setdest -n \<num\_of\_nodes\> -p \<pausetime\> -s \<maxspeed\> -t \<simtime\>
          -x \<maxx\> -y \<maxy\> \> \<outdir\>/\<scenario-file\>
\end{verbatim}
Note that the index used for nodes now start from 0 instead of 1 as
was in the original CMU version, to match with \emph{ns}'s tradition of
assigning node indices from 0.


\subsubsection{Generating traffic pattern files}
\label{sec:mobile-traffic-file}

The examples for traffic patterns may be found in
\textasciitilde\emph{ns}/{tcl/mobility/scene/cbr-50-\{10-4-512, 20-4-512\}}.

The traffic generator is located under \textasciitilde\emph{ns}/{indep-utils/cmu-scen-gen/}
and are called cbrgen.tcl and tcpgen.tcl. They may be used for
generating CBR and TCP connections respectively.

To create CBR connecions, run
\begin{verbatim}
ns cbrgen.tcl [-type cbr|tcp] [-nn nodes] [-seed seed] 
              [-mc connections] [-rate rate]
\end{verbatim}
To create TCP connections, run
\begin{verbatim}
ns tcpgen.tcl [-nn nodes] [-seed seed]
\end{verbatim}
You will need to pipe the outputs from above to a cbr-* or a tcp-* file.


\section{Extensions made to CMU's wireless model}
\label{sec:wireless-extensions}

As mentioned earlier, the original CMU wireless model allows
simulation of wireless LANs and ad-hoc networks. However in order to
use the wireless model for simulations using both wired and wireless
nodes we had to add certain extensions to cmu model. We 
call this wired-cum-wireless feature. Also SUN's MobileIP (implemented
for wired nodes) was integrated into the wireless model allowing
mobileIP to run over wireless mobilenodes. The following two
subsections describe these two extensions to the wireless 
model in \emph{ns}. 


\subsection{wired-cum-wireless scenarios}
\label{sec:wired-cum-wireless}

The mobilenodes described so far mainly supports simulation of
multi-hop ad-hoc networks or wireless LANs. But what if we need to
simulate a topology of multiple wireless LANs connected through wired
nodes, or may need to run mobileIP on top of these wireless nodes? The
extensions made to the CMU wireless model allows us to do that. 

The main problem facing the wired-cum-wireless scenario was the issue
of routing. In ns, routing information is generated based on the
connectivity of the topology, i.e how nodes are connected to one
another through {\tt Links}. Mobilenodes on the other hand have no
concept of links. They route packets among themselves, within the
wireless topology, using their routing protocol. so how would packets
be exchanged between these two types of nodes? 

So a node called {\tt BaseStationNode} is created which plays the
role of a gateway for the wired and wireless domains. The
{\tt BaseStationNode} is essentially a hybrid between a Hierarchical
node\footnote{Refer to Chapter~\ref{chap:hier-rtg} for details on
  hierarchical routing and internals of {\tt HierNode}.}
({\tt HierNode}) and a {\tt MobileNode}. The basestation node is
responsible for delivering packets into and out of the wireless
domain. In order to achieve this we need Hierarchical routing. 

Each wireless domain along with its base-station would have an unique
domain address assigned to them. All packets destined to a wireless
node would reach the base-station attached to the domain of that
wireless node, who would eventually hand the packet
over to the destination (mobilenode). And mobilenodes route packets,
destined to outside their (wireless) domain, to their base-station
node. The base-station knows how to forward these packets towards the
(wired) destination.
\begin{figure}
    \centerline{\includegraphics{basestation}}
    \caption{Schematic of a baseStationNode}
    \label{fig:mobilenode-basestation}
\end{figure}
The schematic of a {\tt BaseStationNode} is shown in
Figure~\ref{fig:mobilenode-basestation}.

The mobilenodes in wired-cum-wireless scenario are required to support
hierarchical addressing/routing. Thus the {\tt MobileNode} looks
exactly like the {\tt BaseStationNode}. The SRNode, however, simply
needs to have its own hier-address since it does not require any
address demuxes and thus is not required to support hier
routing\footnote{In order to do away with all these different
  variations of the definition of a node, we are planning to revise
  the node architecture that would allow a more flexible
  and modularised construction of a node without the necessity of having
  to define and be limited to certain Class definitions only.}.

The DSDV agent on having to forward a packet checks to see if the
destination is outside its (wireless) subnet. If so, it tries to
forward the packet to its base-station node. In case no route to
base-station is found the packet is dropped. Otherwise the
packet is forwarded to the next\_hop towards the base-station. Which
is then routed towards the wired network by base-station's
classifiers.

The DSR agent, on receiving a pkt destined outside its subnet, sends
out a route-query for its base-station in case the route to
base-station is not known. The data pkt is temporarily cached while it
waits to hear route replies from base-station. On getting a reply the
packet is provided with routing information in its header and send
away towards the base-station. The base-station address demuxes routes
it correctly toward the wired network.

The example script for a wired-cum-wireless simulation can be found at
\textasciitilde\emph{ns}/{tcl/ex/wired-cum-wireless-sim.tcl}. The methods for
wired-cum-wireless implementations are defined in
\textasciitilde\emph{ns}/{tcl/lib/ns-bsnode.tcl}, \textasciitilde\emph{ns}/{tcl/mobility/\{com.tcl,dsr.tcl,
  dsdv.tcl\}}, \textasciitilde\emph{ns}/{dsdv/dsdv.\{cc,h\}} and
\textasciitilde\emph{ns}/{dsr/dsragent.\{cc,h\}}.


\subsection{MobileIP}
\label{sec:mobileip}

The wired-cum-wireless extensions for the wireless model paved the
path for supporting wireless MobileIP in \emph{ns}. Sun Microsystem's
(Charlie Perkins {\em et al}) MobileIP model was based on ns's wired model
(consisting of {\tt Node}'s and {\tt Link}'s) and thus didnot use
CMU's mobility model.

Here we briefly describe the wireless MobileIP implementation. We hope
that Sun would provide the detailed version of the documentation in
the future.

The mobileIP scenario consists of Home-Agents(HA) and
Foreign-Agents(FA) and have Mobile-Hosts(MH) moving between their HA
and FAs.
The HA and FA are essentially base-station nodes we have described
earlier. While MHs are basically the mobileNodes described in
section~\ref{sec:mobilenode-creation}.
The methods and procedures for MobileIP extensions are described in
\textasciitilde\emph{ns}/{mip.\{cc,h\}}, \textasciitilde\emph{ns}/{mip-reg.cc}, \textasciitilde\emph{ns}/{tcl/lib/ns-mip.tcl} and
\textasciitilde\emph{ns}/{tcl/lib/ns-wireless-mip.tcl}.

\begin{figure}
    \centerline{\includegraphics{wireless-mip}}
    \caption{Schematic of a Wireless MobileIP BaseStation Node}
    \label{fig:mobilenode-wireless-mip}
\end{figure}
The HA and FA nodes are defined as {\tt MobileNode/MIPBS} having a
registering agent (regagent\_) that sends beacon out to the
mobilenodes, sets up encapsulator and decapsulator, as required and
replies to solicitations from MHs. 
The MH nodes are defined as {\tt MobileNode/MIPMH} which too have a
regagent\_ that receives and responds to beacons and sends out
solicitations to HA or FAs. Figure~\ref{fig:mobilenode-wireless-mip}
illustrates the schematic of a {\tt MobileNode/MIPBS} 
node. The {\tt MobileNode/MIPMH} node is very similar to this except
for the fact that it doesnot have any encapsulator or decapsulator. As
for the SRNode version of a MH, it doesnot have the hierarchical
classifiers and the RA agent forms the entry point of the node. See
Figure~\ref{fig:mobilenode-dsr} for model of a SRNode. 

The {\tt MobileNode/MIPBS} node routinely broadcasts beacon or
advertisement messages out to MHs. A solicitation from a mobilenode
generates an ad that is send directly to the requesting MH. The
address of the base-station sending out beacon is heard by 
MH and is used as the COA (care-of-address) of the MH. Thus as the MH
moves from its native to foreign domains, its COA changes. 
Upon receiving  reg\_request (as reply to ads) from a mobilehost the
base-station checks to see if it is the HA for the MH. If not, it sets
up its decapsulator and forwards the reg\_request towards the HA of
the MH. 

In case the base-station {\em is} the HA for the requesting MH but the
COA doesnot match its own, it sets up an encapsulator and sends
reg-request-reply back to the COA (address of the FA) who has
forwarded the reg\_request to it. so now all packets destined to the
MH reaching the HA would be tunneled through the encapsulator which
encapsulates the IP pkthdr with a IPinIP hdr, now destined to the COA
instead of MH. The FA's decapsulator recives this packet, removes the
encapsulation and sends it to the MH.

If the COA matches that of the HA, it just removes the encapsulator it
might have set up (when its mobilehost was roaming into foreign
networks) and sends the reply directly back to the MH, as the MH have
now returned to its native domain.

The mobilehost sends out solicitations if it doesnot hear any ads from the
base-stations. Upon receiving ads, it changes its COA to the address of
the HA/FA it has heard the ad from, and replies back to the COA with a
request for registration ({\tt reg-request}).
Initially the MH maybe in the range of the HA and receives all pkts
directly from its COA which is HA in this case. 
Eventually as the MH moves out of range of its HA and into the a foreign
domain of a FA, the MH's COA changes from its HA to that of the FA. The HA
now sets up an encapsulator and tunnels all pkts destined for MH towards
the FA. The FA decapsulates the pkts and hands them over to the MH. The
data from MH destined for the wired world is always routed towards its
current COA.  
An example script for wireless mobileIP can be found at
\textasciitilde\emph{ns}/{tcl/ex/wireless-mip-test.tcl}. The simulation consists of a MH moving
between its HA and a FA. The HA and FA are each connected to a wired
domain on one side and to their wireless domains on the other. TCP flows
are set up between the MH and a wired node. 

\section{802.11 MAC protocol}
\label{sec:802_11}
Prior to release ns-2.33, there was only one main-tree 802.11 model,
although other researchers were maintaining third-party patches
on the web.  Starting with ns-2.33, there are multiple choices in
the main distribution.

The first extension described below (infrastructure mode) extends
the legacy model to include infrastructure mode.  However, the last
two items (802.11Ext and dei802mr) are complete replacements for the
legacy model.

Therefore, researchers now have a choice of 802.11 models, and
should carefully read the documentation and code of each one to
understand which is the best fit for the job. 

\begin{description}
\item[{\bf 802.11 DCF from CMU}] 
  This model has been the only model available in the main \emph{ns} source
  tree prior to release ns-2.33.  See \textasciitilde\emph{ns}/{mac-802\_11.\{cc,h\}} for 
  implementation details.  It uses a 
  RTS/CTS/DATA/ACK pattern for all unicast packets and simply sends out
  DATA for all broadcast packets. The implementation uses both 
  physical and virtual carrier sense. The
  \clsref{Mac802\_11}{../ns-2/mac-802\_11.h} is implemented in
  \textasciitilde\emph{ns}/{mac-802\_11.\{cc,h\}}.
\item[{\bf 802.11 infrastructure extensions}] 
  Ilango Purushothaman from the University of Washington has implemented
  infrastructure extensions to the above 802.11 model, and fixed some bugs
  along the way.  The extensions include passive and active scanning,
  authentication, association, inter-AP communications, and mobility
  support (handoff).  Please note that this model still supports
  single-channel scenarios only. 
\begin{itemize}
\item {\bf Documentation:} http://ee.washington.edu/research/funlab/802\_11/report\_80211\_IM.pdf
\item {\bf Example script:} tcl/ex/infra.tcl 
\item {\bf Test suite:} tcl/test/test-suite-wireless-infra-mobility.tcl tcl/test/test-suite-wireless-infra.tcl
\end{itemize}
\item[{\bf 802.11Ext}]  
  A team from Mercedes-Benz Research and Development North America and
  from University of Karlsruhe have collaborated to develop a completely
  new 802.11 Mac and Phy model, called Mac802\_11Ext and WirelessPhyExt,
  respectively.  The new model contains the following features:
\begin{itemize}
\item Structured design of MAC functionality modules:  transmission, 
reception, transmission coordination, reception coordination, backoff
manager, and channel state monitor
\item Cumulative SINR computation
\item MAC frame capture capabbilities
\item Multiple modulation scheme support
\item Packet drop tracing at the PHY layer
\item Nakagami fading model
\end{itemize}
This model should be used as a replacement for the existing models.  The
example scripts show how to do this.
\begin{itemize}
\item {\bf Key files:}  apps/pbc.\{cc,h\}, mac/mac-802\_11Ext.\{cc,h\}, mac/wireless-phyExt.\{cc,h\}, mobile/nakagami.\{cc,h\}
\item {\bf Documentation:}  http://dsn.tm.uni-karlsruhe.de/Overhaul\_NS-2.php
\item {\bf Example scripts:}  tcl/ex/802.11/ directory:  IEEE802-11a.tcl IEEE802-11p.tcl broadcast\_validation.tcl unicast\_validation.tcl
\item {\bf Test suite:}  tcl/test/test-suite-wireless-lan-newnode-80211Ext.tcl
\end{itemize}
\item[{\bf dei80211mr}]
The dei80211mr library - nicknamed 'multirate' for short - provides an 
802.11 derived from the CMU implementation.
This library depends on the Dynamic Library (Chapter \ref{chap:dynlib}) and
is included in the ns-allinone distribution only (see the 
top-level dei80211mr directory in the ns-allinone distribution or
see http://www.dei.unipd.it/wdyn/?IDsezione=5091).
For step-by-step installation instructions, please refer to the tutorial at
http://www.dei.unipd.it/\%7Ebaldo/nsmiracle-dei80211mr-howto.html
The following functionalities are provided by the dei80211mr library:
\begin{itemize}
\item support for multiple PHY modes is included; in particolar, dei80211mr simulation of the different transmission rates, modulation and coding schemes defined in the IEEE802.11b/g standards.
\item a SINR-based packet level error model is introduced:
\begin{itemize}
          \item the RX Threshold variable which was used in the 802.11 implementation included in standard NS to determine successful receptions has been removed. Instead, Packet Error Rate (PER) is used to determine random packet losses.
          \item PER is calculated using pre-determined curves (PER vs SINR and packet size); the curves can be specified by the user via TCL. Some default curves for both 802.11g and 802.11b are provided.
          \item SINR is calculated using received signal strength, noise and interference
          \item interference is calculated using a gaussian model to account for all transmissions which happen simultaneously to the one which is considered for reception
          \item noise power is set via TCL
\end{itemize}
    \item the capture model, i.e. the determination of whether a packet can be received when there are other concurrent transmissions are simultaneously ogoing, is now embedded in the above mentioned interference model (no more Capture Threshold)
    \item In the wireless channel, the affected nodes distance is no more determined using the CS threshold, but we used a fixed value in meters 
which can be set at the beginning of the simulation. The reason is that, since we use a gaussian interference model, nodes well below the CS threshold often still provide a non-negligible contribution to interference. The default value for the affected nodes distance 
is very conservative, so that all nodes are considered for interference calculation. This default value therefore yields accurate but computationally intensive simulations. The value can be adjusted via TCL to achieve different trade-offs between computational load and simulation accuracy.
\end{itemize}
\begin{itemize}
\item {\bf Documentation:} 
http://www.dei.unipd.it/\%7Ebaldo/nsmiracle-dei80211mr-howto.html
\item {\bf Example script:} dei80211mr-1.1.4/samples/adhoc\_tcp.tcl
\item {\bf Test suite:} None
\end{itemize}
\end{description}
  
In addition, a  patch (relating to the CMU implementation) improving ns-2 
802.11 wireless support is available at http://www.telematica.polito.it/fiore/.
The patch introduces realistic channel propagation, concurrent multiple 
data transmission rates among stations and ARF mechanisms, has been 
tested with ns-2.29, and features the following contributions:
\begin{itemize}
\item channel propagation improvements by Wu Xiuchao
\item ricean propagation model by Ratish J. Punnoose
\item SNOOPy calendar scheduler by David X. Wei
\item 802.11 bug fixes by Felix Schmidt-Eisenlohr
\item multiple data transmission rates support by Marco Fiore
\item Adaptive Auto Rate Fallback (AARF) by Marco Fiore. 
\end{itemize}

\section{Lists of changes for merging code developed in older version of ns (2.1b5 or later) into the current version (2.1b8) }
\label{old-merge}

The CMU-wireless model developed by David Johnhson's Monarch project was merged into ns around 1998-99 in what was then the ns-2.1b5 version. Since then the ns versions used by Monarch and by us here at ISI have forked quite a bit. Recently we ported a newer version of DSR developed by the Monarch group back into ns and in the process have created a list of changes that were required to be made for the merge. Hopefully this list will be helpful for those who have been working on older versions of ns from around that time or or later, to have their stuff merged in to the current version of ns-2.1b8. 

The following lists of changes are required for merging the cmu version of ns (2.1b5) in to current version of 2.1b8. Each change is followed by a brief explanation for why the change was made.

\begin{flushleft}

Methods for accessing pkt hdrs have changed from \\
{\tt (hdr\_sr *)p-\>access(off\_sr) } \\
to a static access method defined for each hdr, as \\
{\tt hdr\_sr::access(p)} \\
where for class hdr\_sr a static method {\tt access()} is defined as
\begin{verbatim}
 inline static hdr\_sr* access(const Packet* p) {
		 return (hdr\_sr*)p-\>access(offset\_); 
 }
\end{verbatim}
{\tt why:} This change avoids using casts everywhere.


As the method for accessing hdrs have changed, there is no need to explicitly bind the hdr offset values. This is now done while establishing tcl linkage for the individual hdr classes.
so lines like \\
{\tt bind("off\_SR\_", \&off\_sr\_);}\\	
  {\tt bind("off\_ll\_", \&off\_ll\_);}\\
  {\tt bind("off\_mac\_", \&off\_mac\_);}\\
  {\tt bind("off\_ip\_", \&off\_ip\_); }\\
should be removed.

AF\_ enumerations replaced by NS\_AF\_ as in \\
{\tt enum ns\_af\_enum { NS\_AF\_NONE, NS\_AF\_ILINK, NS\_AF\_INET };}\\
{\tt why:} This avoids header clashes between ns and the OS.


The ip hdr (dst/src) address fields that used be integers are now defined as structures called ns\_addr\_t. ns\_addr\_t has 2 members address\_ and port\_ that are both defined as int. 
Hence lines like \\
{\tt iph-\>src()} should change to\\
{\tt iph-\>saddr() \& iph-\>sport();} \\
Also lines like \\
{\tt dst\_ = (IP\_BROADCAST \<\< 8) | RT\_PORT } \\
should be replaced by \\
{\tt  dst\_.addr\_ = IP\_BROADCAST;} \\
{\tt dst\_.port\_ = RT\_PORT;} \\
{\tt Why:} This extension supports 32bit addressing.


The addrs\_ member for hdr\_sr class has a separate function for returning its value . Thus need to call {\tt hsr.addrs()} instead of hsr.addrs.\\
{\tt why:} addrs\_ is now a private variable which is accessed by public function {\tt addrs()}.


All includes that had absolute paths by using {\tt \<\>} were replaced by {\tt ""}. Thus\\
{\tt \<cmu/dsr/dsragent.h\>}\\
was changed to\\
{\tt "cmu/dsr/dsragent.h"}


The tcl command "ip-addr" was changed to "addr".\\
Other new tcl commands like "node", "port-dmux" and "trace-target" were added.\\
{\tt why:}  Part of support for mobileIP and wired-cum-wireless simulations.


Need to convert address in string format into int format; 
so use\\
{\tt Address::instance().str2addr(argv[2]) }\\
instead of \\
{\tt atoi(argv[2])}\\
{\tt why:} This is required for supporting hier-addressing/routing.


The array {\tt packet\_names[]} has changed to {\tt packet\_info.name()}\\
{\tt why:} In order to remove a bunch of {\tt #}defines for pkt types, an enumeration called packet\_t now describes all packet types in ns. class p\_info was created that now describes an array name\_ that has replaced packet\_names array used previously.


Have to explicitly set direction of new pkts to DOWN before sending them down to the LL.\\
{\tt why:} A variable direction\_ in hdr\_cmn is now used. This is used in the lower layers like LL, mac, phy etc to determine the direction of the pkt flow. All incoming pkts are marked as UP by channel, which should be remarked as DOWN by agents before sending them out into the network again.


Instead of {\tt logtarget-\>buffer}, should now call {\tt logtarget-\>pt\_-\>buffer}.\\
{\tt why:} This change reflects support for eventtracing. 
Tracing has evolved into two types, packet tracing and event tracing.
Class Trace essentially supports packet tracing. 
However in addition to the basic tracing properties that it derives from a BaseTrace class, pkt-tracing also requires to inherit some of the Connector class properties as well. Hence pt\_, a basetrace object represents the pure tracing functionalities required for a trace object.


The parameter used to describe the reason a pkt was dropped used to be an integer. This was changed to {\tt char*}. Hence needed to define different pkt-drop reasons in string formats. \\
{\tt Why:} Allows greater expandibility and flexibility.


linkHead changed to dsrLinkHead.\\
{\tt why:} name clashed with linkHead used elsewhere in ns.


The older cmu model used an incoming\_ flag added in all pkts to figure out direction of pkt flow in the lower layers like ll, mac etc. Later this was replaced by a variable called direction\_ added in cmn\_hdr. direction value can be set to UP, DOWN or NONE. all pkts created with a DOWN dir by default. \\
{\tt why:} Both these flags were being used which is not really reqd. so incoming\_ flag has been replaced with direction\_.

\end{flushleft}


\section{Commands at a glance}
\label{sec:wirelesscommand}

Following is a list of commands used in wireless simulations:
\begin{verbatim}
\$ns\_ node-config -addressingType \<usually flat or hierarchical used for 
                                   wireless topologies\>
                 -adhocRouting   \<adhoc rotuing protocol like PUMA, DSR,
                                   TORA, AODV, DSDV etc\>
                 -llType         \<LinkLayer\>
                 -macType        \<MAC type like Mac/802\_11\>
                 -propType       \<Propagation model like
                                   Propagation/TwoRayGround\>
                 -ifqType        \<interface queue type like
                                   Queue/DropTail/PriQueue\>
                 -ifqLen         \<interface queue length like 50\>
                 -phyType        \<network inteface type like
                                   Phy/WirelessPhy\>
                 -antType        \<antenna type like Antenna/OmniAntenna\>
                 -channelType    \<Channel type like Channel/WirelessChannel\>
                 -topoInstance   \<the topography instance\>
                 -wiredRouting   \<turning wired routing ON or OFF\>
                 -mobileIP       \<setting the flag for mobileIP ON or OFF\>
                 -energyModel    \<EnergyModel type\>
                 -initialEnergy  \<specified in Joules\>
                 -rxPower        \<specified in W\>
                 -txPower        \<specified in W\>
                 -agentTrace     \<tracing at agent level turned ON or OFF\>
                 -routerTrace    \<tracing at router level turned ON or OFF\>
                 -macTrace       \<tracing at mac level turned ON or OFF\>
                 -movementTrace  \<mobilenode movement logging turned
                                   ON or OFF\>
\end{verbatim}
This command is used typically to configure for a mobilenode. For more info
about this command (part of new node APIs) see chapter titled "Restructuring
ns node and new Node APIs" in ns Notes and Documentation.

\begin{flushleft}
{\tt \$ns\_ node \<optional:hier address\>}\\
This command is used to create a mobilenode after node configuration is done
as shown in the node-config command. Incase hierarchical addressing is being
used, the hier address of the node needs to be passed as well.


{\tt \$node log-movement}\\
This command previously used to enable logging of mobilenode's movement has now
been replaced by {\tt \$ns\_ node-config -movementTrace \<ON or OFF\>}.


{\tt create-god \<num\_nodes\>}\\
This command is used to create a God instance. The number of mobilenodes
is passed as argument which is used by God to create a matrix to store
connectivity information of the topology.


{\tt \$topo load\_flatgrid \<X\> \<Y\> \<optional:res\>}\\
This initializes the grid for the topography object. <X> and <Y> are the x-y
co-ordinates for the topology and are used for sizing the grid. The grid
resolution may be passed as <res>. A default value of 1 is normally used.


{\tt \$topo load\_demfile \<file-descrptor\>}\\
For loading DEMFile objects into topography. See \emph{ns}/dem.{cc,.h} for details on
DEMFiles.


{\tt \$ns\_ namtrace-all-wireless \<namtrace\> \<X\> \<Y\>}\\
This command is used to initialize a namtrace file for logging node movements
to be viewed in nam. The namtrace file descriptor, the X and Y 
co-ordinates of the wireless topology is passed as parameters with
this command.


{\tt \$ns\_ nam-end-wireless \<stop-time\>}\\
This command is used to tell nam the simulation stop time given by <stop-time>.


{\tt \$ns\_ initial\_node\_pos \<node\> \<size\>}\\
This command defines the node initial position in nam. <size> denotes the size
of node in nam. This function must be called after mobility model has been
defined.


{\tt \$mobilenode random-motion \<0 or 1\>}\\
Random-motion is used to turn on random movements for the mobilenode, in which
case random destinations are assigned to the node. 0 disables and 1 enables
random-motion.


{\tt \$mobilenode setdest \<X\> \<Y\> \<s\>}\\
This command is used to setup a destination for the mobilenode. The mobile
node starts moving towards destination given by <X> and <Y> at a speed of
<s> m/s.


{\tt \$mobilenode reset}\\
This command is used to reset all the objects in the nodes (network 
components like LL, MAC, phy etc).


Internal procedures\\
Following is a list of internal procedures used in wireless networking:

{\tt \$mobilenode base-station \<BSnode-hier-addr\>}\\
This is used for wired-cum-wireless scenarios. Here the mobilenode is provided
with the base-stationnode info for its domain. The address is hierarchical
since wired-cum-wireless scenarios typically use hierarchical addressing.


{\tt \$mobilenode log-target  \<target-object\>}\\
The <target-object>, which is normally a trace object, is used to log
mobilenode movements and their energy usage, if energy model is provided.


{\tt \$mobilenode topography \<topoinstance\>}\\
This command is used to provide the node with a handle to the topography
object.


{\tt \$mobilenode addif}\\
A mobilenode may have more than one network interface. This command is used
to pass handle for a network interface to the node.


{\tt \$mobilenode  namattach \<namtracefd\>}\\
This command is used to attach the namtrace file descriptor <namtracefd>
to the mobilenode. All nam traces for the node are then written into this
namtrace file.


{\tt \$mobilenode radius \<r\>}\\
The radius <r> denotes the node's range. All mobilenodes that fall within
the circle of radius <r> with the node at its center are considered as
neighbours. This info is typically used by the gridkeeper.


{\tt \$mobilenode start}\\
This command is used to start off the movement of the mobilenode.

\end{flushleft}

\endinput


